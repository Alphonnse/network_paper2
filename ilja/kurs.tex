\documentclass[14pt, a4paper]{extarticle}
% Русская локализация
\usepackage[english,russian]{babel}

% Использование математических шрифтов
\usepackage{unicode-math}

% Шрифты
\usepackage{fontspec}
\usepackage{courier}
\defaultfontfeatures{Ligatures={TeX},Renderer=Basic}
\setmainfont[Ligatures={TeX}]{Times New Roman}
\setmonofont{Courier New}
\setmathfont{XITS Math}

% Расширенные ссылки
\usepackage{nameref}

% Оформление URL
\usepackage{xurl}
\usepackage{hyperref}
\hypersetup{
    colorlinks,
    citecolor=black,
    filecolor=black,
    linkcolor=black,
    urlcolor=black,
    breaklinks=true,
}
\urlstyle{same}

% Поддержка изображений
\usepackage{graphicx}
\graphicspath{{./images/}}
\DeclareGraphicsExtensions{.jpg,.png}

% Таблицы
\usepackage{tabularx}
\usepackage{tabulary}
\usepackage{multirow}
\usepackage{hhline} 
% Выравнивание по левому краю, с многострочностью
\newcolumntype{s}{>{\raggedright\arraybackslash}X}

% Поддержка листингов
\usepackage{listings}
\lstdefinestyle{gost}{
    basicstyle=\ttfamily\footnotesize,
    breakatwhitespace=false,
    breaklines=true,
    keepspaces=true,
    showspaces=false,          
    showstringspaces=false,
    frame=single
}
\lstset{style=gost}%

% Отступ первой строки первого абзаца
\usepackage{indentfirst}
\linespread{1.25}

% Размер полей в документе
\usepackage{geometry}
\geometry{left=3cm}
\geometry{right=1cm}
\geometry{top=2cm}
\geometry{bottom=2cm}

% Абзацный отступ
\setlength{\parindent}{1.25cm}

% Отступ для элементов в списке
\usepackage{enumitem}
\setlist{left=\parindent, labelsep=1cm, itemsep=0pt, topsep=0pt}

% Загрузка pdf-документов (нужно для титульников)
\usepackage[final]{pdfpages}
% Возможность поворота pdf файло
\usepackage{pdflscape}
\usepackage{everypage}

\newcommand{\Lpagenumber}{\ifdim\textwidth=\linewidth\else\bgroup%
  \dimendef\margin=0 %use \margin instead of \dimen0
  \ifodd\value{page}\margin=\oddsidemargin
  \else\margin=\evensidemargin%
  \fi
  \raisebox{\dimexpr-\topmargin-\headheight-\headsep-0.5\linewidth}[0pt][0pt]{%
    \rlap{\hspace{\dimexpr-\margin+\textheight+\footskip}%
    \llap{\rotatebox{90}{\thepage}}}}%
\egroup\fi}
\AddEverypageHook{\Lpagenumber}%

\usepackage{float}
% Форматирование подписей
\usepackage{caption}

\usepackage{newfloat}
%\DeclareCaptionType{listing}

\DeclareCaptionLabelSeparator{emdash}{\;\textemdash\;}
\captionsetup[figure]{name={Рисунок}, labelsep=emdash, justification=centering, position=above, singlelinecheck=off, font={small, bf}, labelfont=bf, skip=6pt}
\captionsetup[table]{name={Таблица}, labelsep=emdash, justification=raggedright, position=top, singlelinecheck=off, font={small, it}, labelfont=it, skip=6pt, margin=0cm}
\captionsetup[lstlisting]{labelsep=emdash, justification=raggedright, position=top, singlelinecheck=off, font={small, it}, labelfont=it, skip=6pt, margin=0cm}

% Нумеровать внутри заголовков первого уровня
\counterwithin{figure}{section}
\counterwithin{table}{section}
%\counterwithin{lstlisting}{section}
\AtBeginDocument{\counterwithin{lstlisting}{section}}

% Отключение переносов текста
\usepackage{ragged2e}
\justifying
\tolerance=500
\hyphenpenalty=10000
\emergencystretch=3em

% Форматирование заголовков
\usepackage{titlesec}
% Оформление заголовка первого уровня
% Полужирное начертание
% Кегль 18 пт
% С новой страницы
\titleformat{\section}[block]
	{\newpage\bfseries\fontsize{18pt}{21.6pt}\selectfont}
        {\thesection}
        {1em}{}
% Оформление ненумерованных заголовков (Введение, Содержание, список источников, и.т.д.)
\titleformat{name=\section,numberless}[block]
	{\centering\newpage\bfseries\fontsize{18pt}{21.6pt}\selectfont}
        {}
        {0em}{}{}
% Отступы у заголовков первого уровня
\titlespacing{\section}
 {\parindent}% отступ слева (равен 1.25 см, как у отступа первой строки абзаца)
 {0em}% интервал перед
 {10mm}% интервал после
% Оформление заголовков второго уровня
\titleformat{\subsection}[block]
	{\bfseries\fontsize{16pt}{19.2pt}\selectfont}
        {\thesubsection}
        {1em}{}
% Отступы у заголовков второго уровня
\titlespacing{\subsection}
 {\parindent}% пробел слева
 {15mm}% отступ перед
 {10mm}% отступ после
% Оформление заголовков второго уровня
\titleformat{\subsubsection}[block]
	{\bfseries\selectfont}
        {\thesubsection}
        {1em}{}
% Отступы у заголовков второго уровня
\titlespacing{\subsubsection}
 {\parindent}% пробел слева
 {15mm}% отступ перед
 {10mm}% отступ после

% Оформление заголовков в содержании
\usepackage{titletoc}
\contentsmargin{0pt}
\renewcommand\contentspage{\thecontentspage}
\dottedcontents{section}[2.3em]{}{2.3em}{5pt}
\dottedcontents{subsection}[2.3em]{}{2.3em}{5pt}
% Оформление приложений
\usepackage{appendix}
\renewcommand\appendixpagename{ПРИЛОЖЕНИЯ}

% Подключение biblatex, с использованием стиля gost-numeric
\usepackage[
citestyle=gost-numeric,
style=gost-numeric, 
blockpunct=emdash,
backend=biber,
sorting=none
]{biblatex}
% Запрет разрыва url ссылок
\defcounter{biburlnumpenalty}{3000}
\defcounter{biburlucpenalty}{6000}
\defcounter{biburllcpenalty}{9000}
% Добавление полей для ссылок и даты обращения к ним
\DeclareFieldFormat{url}{Режим доступа: #1}
\DeclareFieldFormat{urldate}{(Дата обращения: #1)}
\renewcommand*{\entrysetpunct}{\par\nopunct\!\!}
% Использовать prac.bib как источник
\addbibresource{kurs.bib}
% Форматирование заголовка библиографии
\defbibheading{bibliography}[\bibname]{%
  \section*{\centering #1}%
  \markboth{#1}{#1}}


\usepackage{lipsum}
\begin{document}
\def\contentsname{СОДЕРЖАНИЕ}

% Загрузка титула
\pagenumbering{gobble}
\begin{titlepage}
\includepdf[pages={1-2}]{title}
\end{titlepage}
\pagenumbering{arabic}
\setcounter{page}{3}
% Содержание
\tableofcontents

\clearpage
\section*{ВВЕДЕНИЕ}
\addcontentsline{toc}{section}{ВВЕДЕНИЕ}

В современном мире использование сети Интернет является неотъемлемой
частью жизни общества. Для коммерческого же сектора сеть
является важным инструментом для работы предприятия. Она позволяет
быстро, безопасно и надежно передавать информацию между
организационными единицами предприятия, ускоряет процессы принятия
решений и снижает затраты на обмен информацией. Это делает сеть
передачи данных необходимым инструментом элементом инфраструктуры
предприятий, которая требует постоянного обслуживания и грамотного
проектирования.

В рамках данной курсовой работы будет рассмотрен процесс
проектирования и реализации сети передачи данных на примере
предприятия, которая осущевствляет оптовую торговлю бытовой химией.

Актуальность темы <<Проектирование сети передачи данных предприятия>>
обусловлена необходимостью создания эффективной инфраструктуры для
передачи данных внутри предприятия, которая обеспечит более
эффективное исполнение бизнес-процессов предприятия. Объектом
исследования является сеть передачи предприятия, предметом ---
особенности проектирования и реализации сети передачи данных, целью
--- планирование и реализация сети передачи данных.

Задачи исследования включают в себя анализ исходных данных
поставленной задачи --- изучение изучение требований и ограничений,
связанных с созданием сети, а также анализ имеющейся инфраструктуры и
ресурсов, планирование сетей уровня 2 и 3 --- изучение разделения сети
на подсети, разделения физических сетей на виртуальные логические
сети, определение маршрутизации, прототипирование и настройка сети с
последующим тестированием. А также планирование политик фильтрации
трафика, управления пользователей, сетевых сервисов.

В работе будут использованы такие методы исследования, как анализ
научной литературы, анализ статистических данных, сравнительный анализ
различных технологий и протоколов передачи данных, моделирование и
прототипирование с помощью специальных программных средств.

В качестве инструментальных средств будут использованы:
\begin{enumerate}
\item Программное средство проектирования сети передачи данных Cisco Packet Tracer.
\item Веб-сервис для построения топологии сети и различных диаграмм
  diagrams.net.
\item Средства виртуализации для примерного развертывания сервисов, а
  также расчета нагрузки сетевых сервисов на сеть (Oracle VirtualBox).
\end{enumerate}

Основными источниками будут являться нормативные документы, научные
статьи и книги.

Основное содержание работы будет состоять из следующих пунктов:
\begin{enumerate}
\item Определение структуры предприятия.
\item Расчёт пропускной способности каналов передачи данных.
\item Прототипирование сети.
\item Планирование сети уровня 2.
\item Планирование сети уровня 3.
\item Планирование политик фильтрации трафика.
\item Планирование политик обеспечения качества обслуживания.
\item Планирование службы доменных имен.
\item Планирование политик управления пользователями.
\item Планирование внедрения DHCP-сервера.
\item Планирование других сетевых служб.
\item Определение и расчет сервисной нагрузки.
\item Моделирование сервисов.
\item Моделирование сети передачи данных
\end{enumerate}

\section{ОПРЕДЕЛЕНИЕ СТРУКТУРЫ ПРЕДПРИЯТИЯ}

В данном пункте требуется конкретизировать структуру предприятия,
определить отделы и группы пользователей, а также требования к сети,
которые определяются пользователями.

Схема расчета количества рабочих мест для основного здания
(Рисунок\;\ref{fig:hq_people}), типовых филиалов
(Рисунок\;\ref{fig:filial_people}), типовых складов
(Рисунок\;\ref{fig:warehouse_people}) представлены ниже. Как следует
из схем предполагается, что все филиалы и склады данного предприятия
являются типовыми, то есть имеют одинаковую структуру вне зависимости от
их расположения. Выделение количества АРМ (автоматизированных рабочих
мест) для АХО (административно-хозяйственного отдела) и отдела
безопасности исходит из посменного графика работы данных
работников. Для складов предполагается, что АРМ требуется только главе
текущей бригады работников склада, а также отделу безопасности, в то
время как остальные работники будут использовать беспроводные
устройства для связи, что в рамках данной работы не
рассматривается. По требованиям к сети предполагается, что все
работники, кроме начальства используют линии в 100 Мбит/с, в то время
как руководство предприятия будет использовать линии в 1 Гбит/с в виду
требований к IP-телефонии и видеоконференцсвязи.

\begin{figure}[H]
  \centering
  \includegraphics{Warehouse_People}
  \caption{Количество рабочих мест на типовом складе}
  \label{fig:warehouse_people}
\end{figure}

\begin{figure}[H]
  \centering
  \includegraphics[width=\linewidth]{HQ_People}
  \caption{Количество рабочих мест в основном здании}
  \label{fig:hq_people}
\end{figure}


\begin{figure}[H]
  \centering
  \includegraphics{Filial_People}
  \caption{Количество рабочих мест в типовом филиале}
  \label{fig:filial_people}
\end{figure}

\section{РАСЧЁТ ПРОПУСКНОЙ СПОСОБНОСТИ КАНАЛОВ ПЕРЕДАЧИ ДАННЫХ}
В данном пункте следует произвести расчет пропускной способности
каналов передачи данных с учетом применения трехуровневой архитектуры
сети, состоящей из уровня доступа, агрегации и ядра.

Расчет портов уровня доступа для основного здания
(Таблица\;\ref{tab:hq_access_ports}) исходит из требований к сети
которые были описаны в работе выше.
\begin{table}[H]
  \caption{Расчет портов уровня доступа для основного здания\label{tab:hq_access_ports}}
  \centering
  \small
  \begin{tabularx}{\textwidth}{|s|s|s|s|>{\hsize=1.1\hsize}s|>{\hsize=0.9\hsize}s|}
    \hline
    Название отдела & Количество АРМ & Требования к каналу передачи данных, Мбит/с & Суммарные требования на отдел, Мбит/с & Портов GigabitEthernet & Портов FastEthernet \\ \hline
    Отдел ИТ        & 27             & 100                                         & 2700                                  & 0                      & 27                  \\ \hline
    Бухгалтерия     & 11             & 100                                         & 1100                                  & 0                      & 11                  \\ \hline
    Отдел кадров    & 11             & 100                                         & 1100                                  & 0                      & 11                  \\ \hline
    Отдел закупок   & 11             & 100                                         & 1100                                  & 0                      & 11                  \\ \hline
    Отдел продаж    & 11             & 100                                         & 1100                                  & 0                      & 11                  \\ \hline
    АХО             & 1              & 100                                         & 100                                   & 0                      & 1                   \\ \hline
    Дирекция        & 3              & 1000                                        & 3000                                  & 3                      & 0                   \\ \hline
    Охрана          & 1              & 100                                         & 100                                   & 0                      & 1                   \\ \hline
    Итого           & 76             & \multicolumn{2}{s|}{-}                                                              & 3                      & 73                  \\ \hline
  \end{tabularx}
\end{table}

Для уровня агрегации расчет исходит с учетом резервирования портов
(Таблица\;\ref{tab:hq_agregation_ports}). Также в качестве
коэффициента перехода объема трафика на уровень агрегации для всех
зданий было взято 40\%, так как узлы не будут постоянно использовать
весь канал передачи данных из-за того, что основным сервисом является
программное обеспечение 1С: Предприятие, которое не требует особых
требований к постоянному широкополосному каналу передачи
данных. Количество портов на уровне ядра будет совпадать с уровнем
агрегации.

\begin{table}[H]
  \caption{Расчет портов уровня агрегации для основного здания\label{tab:hq_agregation_ports}}
  \centering
  \small
  \begin{tabularx}{\textwidth}{|s|s|s|s|s|s|}
    \hline
    Название отдела & Суммарные требования на отдел, Мбит/с & Объем трафика при переходе на уровень агрегации & Количество портов на уровне агрегации, GigabitEthernet \\ \hline
    Отдел ИТ        & 2700                                  & 1080                                            & 2 + 2 \\ \hline
    Бухгалтерия     & 1100                                  & 440                                             & 1 + 1 \\ \hline
    Отдел кадров    & 1100                                  & 440                                             & 1 + 1 \\ \hline
    Отдел закупок   & 1100                                  & 440                                             & 1 + 1 \\ \hline
    Отдел продаж    & 1100                                  & 440                                             & 1 + 1 \\ \hline
    АХО             & 100                                   & 40                                              & 1 + 1 \\ \hline
    Дирекция        & 3000                                  & 1200                                            & 2 + 2 \\ \hline
    Охрана          & 100                                   & 40                                              & 1 + 1 \\ \hline
    Итого           & -                                     & 4120                                            & 20 \\ \hline
  \end{tabularx}
\end{table}

Расчет портов уровня доступа и агрегации для типовых филиалов выполнен
аналогично основному зданию и представлен на
Таблицах\;\ref{tab:filial_access_ports}~---~\ref{tab:filial_agregation_ports}. Главным
отличием является факт отсутствия отдела ИТ на данных
площадках. Требования по пропускной способности, а также коэффициент
перехода на уровень агрегации и уровень резервирования остаются теми
же.

\begin{table}[H]
  \caption{Расчет портов уровня доступа для филиала\label{tab:filial_access_ports}}
  \centering
  \small
  \begin{tabularx}{\textwidth}{|s|s|s|s|>{\hsize=1.1\hsize}s|>{\hsize=0.9\hsize}s|}
    \hline
    Название отдела & Количество АРМ & Требования к каналу передачи данных, Мбит/с & Суммарные требования на отдел, Мбит/с & Портов GigabitEthernet & Портов FastEthernet \\ \hline
    Бухгалтерия     & 5              & 100                                         & 500                                   & 1                      & 5                   \\ \hline
    Отдел кадров    & 2              & 100                                         & 200                                   & 1                      & 2                   \\ \hline
    Отдел закупок   & 9              & 100                                         & 900                                   & 1                      & 9                   \\ \hline
    Отдел продаж    & 10             & 100                                         & 1000                                  & 1                      & 10                  \\ \hline
    АХО             & 1              & 100                                         & 100                                   & 1                      & 1                   \\ \hline
    Дирекция        & 3              & 1000                                        & 3000                                  & 3                      & 30                  \\ \hline
    Итого           & 30             & \multicolumn{2}{s|}{-}                                                              & 8                      & 57                  \\ \hline
  \end{tabularx}
\end{table}

\begin{table}[H]
  \caption{Расчет портов уровня агрегации для филиала\label{tab:filial_agregation_ports}}
  \centering
  \small
  \begin{tabularx}{\textwidth}{|s|s|s|s|s|s|}
    \hline
    Название отдела & Суммарные требования на отдел, Мбит/с & Объем трафика при переходе на уровень агрегации & Количество портов на уровне агрегации, GigabitEthernet \\ \hline
    Бухгалтерия     & 500                                   & 200                                             & 1 + 1 \\ \hline
    Отдел кадров    & 200                                   & 80                                              & 1 + 1 \\ \hline
    Отдел закупок   & 900                                   & 360                                             & 1 + 1 \\ \hline
    Отдел продаж    & 1000                                  & 440                                             & 1 + 1 \\ \hline
    АХО             & 100                                   & 40                                              & 1 + 1 \\ \hline
    Дирекция        & 3000                                  & 1200                                            & 2 + 2 \\ \hline
    Итого           & -                                     & 2320                                            & 14 \\ \hline
  \end{tabularx}
\end{table}

Для типовых складов использование трехуровневой архитектуры сети
нецелесообразно в виду крайне малого количества конечных устройств (их
всего 5). Поэтому расчет пропускной способности сети будет идти без
перехода на уровень агрегации. Требуемая пропускная способность
типового склада представлена в Формуле\;\ref{eq:warehouse_capacity}.
\begin{equation}\label{eq:warehouse_capacity}
  p = 5 * 100 \frac{\text{Мбит}}{\text{c}} = 500 \frac{\text{Мбит}}{\text{с}}.
\end{equation}

Таким образом требуется 5 портов FastEthernet для обеспечения
пропускной способности сети типового склада. Поскольку на складах не
используется трехуровневая архитектура сети, то скорее всего будет
использоваться один маршрутизатор, который будет подключен ко всем
конечным устройствам пользователей. Поэтому использование
резервирования в данном случае нецелесообразно и невозможно, поскольку
в рамках программного обеспечения Cisco Packet Tracer можно подключить
только один сетевой порт на персональный компьютер.

\section{ПРОТОТИПИРОВАНИЕ СЕТИ}
В данном шаге требуется создать прототип сети с учетом
предварительного планирования количество портов, политик
резервирования, добавления сервера для развертывания программной
инфраструктуры предприятия, трехуровневой архитектуры сети, полученных
в предыдущем пункте работы.

Спецификация промежуточных устройств прототипа сети исходит из
доступных устройств в Cisco Packet Tracer. Спецификация для основного
офиса представлена на Таблице\;\ref{tab:hq_device_spec}. В данном
случае используется два маршрутизатора для дополнительного
резервирования подключения к провайдеру сети. Также для соблюдения
требований по пропускной способности сети для руководства будет
использоваться коммутатор L3 из-за недостаточного количества портов
GigabitEthernet на доступных L2 коммутаторах.

Помимо руководства L3 коммутатор потребуется для сервера. Данное
решение позволит соблюдать трехуровневую архитектуру сети, а также
даст возможности для масштабирования сети (можно будет подключать
сразу несколько серверов).

Для остальных конечных устройств L2 коммутаторов будет
достаточно. Количество требуемых коммутаторов L2 рассчитано в
Формуле\;\ref{eq:l2_comm_count}.

\begin{equation}
  \label{eq:l2_comm_count}
  p = \frac{73 \text{порта}}{24 \text{порта}} \approx 3.042
\end{equation}

Таким образом потребовалось бы 4 коммутатора L2 для подключения всех
конечных устройств пользователей, однако на L3 коммутаторе, который
подключен к руководству, имеется достаточно свободных портов для
подключения конечных устройств, поэтому можно воспользоваться тремя
коммутаторами L2.

\begin{landscape}\pagestyle{empty}\newpage
\begin{table}[H]
  \caption{Спецификация промежуточных устройств прототипа сети основного здания\label{tab:hq_device_spec}}
  \centering
  \small
  \begin{tabulary}{\linewidth}{|L|L|L|L|L|L|}
    \hline
    Модель устройства       & Имя устройства                                     & Модули расширения & Общее количество портов (с модулями расширения) & Используемые порты (названия, количество)                        & Свободные порты (названия количество) \\ \hline
    Маршрутизатор 4331      & R\allowbreak\_1\allowbreak\_IVANOV                 & -                 & 3 порта GigabitEthernet                         & 3 порта: GigabitEthernet0/0/0-0/0/2                              & 0 портов \\ \hline
    Маршрутизатор 4331      & R\allowbreak\_2\allowbreak\_IVANOV                 & -                 & 3 порта GigabitEthernet                         & 3 порта: GigabitEthernet0/0/0-0/0/2                              & 0 портов \\ \hline
    Коммутатор L3 3650 24PS & SW\allowbreak\_1\allowbreak\_L3\allowbreak\_IVANOV & -                 & 24 порта GigabitEthernet                        & 10 портов: GigabitEthernet1/0/1-1/0/10                           & 14 портов: GigabitEthernet1/0/11-1/0/24 \\ \hline
    Коммутатор L3 3650 24PS & SW\allowbreak\_2\allowbreak\_L3\allowbreak\_IVANOV & -                 & 24 порта GigabitEthernet                        & 10 портов: GigabitEthernet1/0/1-1/0/10                           & 14 порта: GigabitEthernet1/0/11-1/0/24 \\ \hline
    Коммутатор 2960         & SW\allowbreak\_1\allowbreak\_L2\allowbreak\_IVANOV & -                 & 24 порта FastEthernet, 2 порта GigabitEthernet  & 24 порта: FastEthernet0/1-0/24, 2 порта: GigabitEthernet0/1-0/2  & 0 портов \\ \hline
    Коммутатор 2960         & SW\allowbreak\_2\allowbreak\_L2\allowbreak\_IVANOV & -                 & 24 порта FastEthernet, 2 порта GigabitEthernet  & 13 портов: FastEthernet0/1-0/13, 2 порта: GigabitEthernet0/1-0/2 & 11 портов: FastEthernet0/14-0/24 \\ \hline
    Коммутатор L3 3650 24PS & SW\allowbreak\_3\allowbreak\_L2\allowbreak\_IVANOV & -                 & 24 порта GigabitEthernet                        & 23 порта: GigabitEthernet1/0/1-1/0/23                            & 1 порт: GigabitEthernet1/0/24 \\ \hline
    Коммутатор 2960         & SW\allowbreak\_4\allowbreak\_L2\allowbreak\_IVANOV & -                 & 24 порта FastEthernet, 2 порта GigabitEthernet  & 22 порта: FastEthernet0/1-0/22, 2 порта: GigabitEthernet0/1-0/2  & 2 порта: FastEthernet0/23-0/24 \\ \hline
    Коммутатор L3 3650 24PS & SW\allowbreak\_5\allowbreak\_L2\allowbreak\_IVANOV & -                 & 24 порта GigabitEthernet                        & 3 порта: GigabitEthernet1/0/1-1/0/3                              & 21 порт: GigabitEthernet1/0/4-1/0/24 \\ \hline
  \end{tabulary}
\end{table}
\end{landscape}\pagestyle{plain}\newpage

При построении логической топологии сети основного здания
(Рисунок\;\ref{fig:hq_topology}) требования к пропускной способности
сервера были взяты как 1 Гбит/с, однако расчет реальных требований
будет выполнен далее. Также в качестве типа подключения между
коммутаторами L3 использовалась топология <<кольцо>>. Для соединения между коммутаторами L2 от руководства компании сделаны 3 линии GigabitEthernet (2 для соблюдения пропускной способности, 1 --- резервная).
\begin{figure}[H]
  \centering
  \includegraphics[width=\linewidth]{Topology_HQ}
  \caption{Топология сети основного здания}
  \label{fig:hq_topology}
\end{figure}

План подключений оборудования по портам для основного здания
представлен на Таблице\;\ref{tab:hq_connection_plan}.

\begin{table}[H]
  \caption{План подключений оборудования по портам для основного здания\label{tab:hq_connection_plan}}
  \centering
  \small
  \begin{tabularx}{\textwidth}{|s|s|s|}
    \hline
    Название устройства & Порт                         & Описание подключения                      \\ \hline
    R\_1\_IVANOV        & GigabitEthernet0/0/0         & R\_2\_IVANOV                              \\ \cline{2-3}
                        & GigabitEthernet0/0/1         & SW\_1\_L3\_IVANOV                         \\ \cline{2-3}
                        & GigabitEthernet0/0/2         & SW\_2\_L3\_IVANOV                         \\ \hline
    R\_2\_IVANOV        & GigabitEthernet0/0/0         & R\_1\_IVANOV                              \\ \cline{2-3}
                        & GigabitEthernet0/0/1         & SW\_2\_L3\_IVANOV                         \\ \cline{2-3}
                        & GigabitEthernet0/0/2         & SW\_1\_L3\_IVANOV                         \\ \hline
    SW\_1\_L3\_IVANOV   & GigabitEthernet1/0/1-1/0/7   & SW\_1-5\_L2\_IVANOV                       \\ \cline{2-3}
                        & GigabitEthernet1/0/8         & SW\_2\_L3\_IVANOV                         \\ \cline{2-3}
                        & GigabitEthernet1/0/9-1/0/10  & R\_1-2\_IVANOV                            \\ \hline
    SW\_2\_L3\_IVANOV   & GigabitEthernet1/0/1-1/0/7   & SW\_1-5\_L2\_IVANOV                       \\ \cline{2-3}
                        & GigabitEthernet1/0/8         & SW\_1\_L3\_IVANOV                         \\ \cline{2-3}
                        & GigabitEthernet1/0/9-1/0/10  & R\_1-2\_IVANOV                            \\ \hline
    SW\_1\_L2\_IVANOV   & FastEthernet0/1-0/24         & PC\_1-24\_ИТ\_IVANOV                      \\ \cline{2-3}
                        & GigabitEthernet0/1-0/2       & SW\_1-2\_L3\_IVANOV                       \\ \hline
    SW\_2\_L2\_IVANOV   & FastEthernet0/1-0/13         & PC\_1\_BUKH\_IVANOV - PC\_1\_SEC\_IVANOV  \\ \cline{2-3}
                        & GigabitEthernet0/1-0/2       & SW\_1-2\_L3\_IVANOV                       \\ \hline
    SW\_3\_L2\_IVANOV   & GigabitEthernet1/0/1-1/0/17  & PC\_1\_HR\_IVANOV - PC\_3\_RULE\_IVANOV   \\ \cline{2-3}
                        & GigabitEthernet1/0/18-1/0/21 & SW\_1-2\_L3\_IVANOV                       \\ \hline
    SW\_4\_L2\_IVANOV   & FastEthernet0/1-0/22         & PC\_1\_SELL\_IVANOV - PC\_11\_BUY\_IVANOV \\ \cline{2-3}
                        & GigabitEthernet0/1-0/2       & SW\_1-2\_L3\_IVANOV                       \\ \hline
    SW\_5\_L2\_IVANOV   & GigabitEthernet1/0/1         & SERV\_1\_IVANOV                           \\ \cline{2-3}
                        & GigabitEthernet1/0/2-1/0/3   & SW\_1-2\_L3\_IVANOV                       \\ \hline
  \end{tabularx}
\end{table}


Проектирование логической топологии устройства
(Рисунок\;\ref{fig:filial_topology}) аналогично топологии основного
здания. Спецификация промежуточных устройств, а также план подключения
оборудования по портам представлены на
Таблицах\;\ref{tab:filial_device_spec}~---~\ref{tab:filial_connection_plan}.

\begin{landscape}\pagestyle{empty}\newpage
\begin{table}[H]
  \caption{Спецификация промежуточных устройств прототипа сети типового филиала\label{tab:filial_device_spec}}
  \centering
  \small
  \begin{tabulary}{\linewidth}{|L|L|L|L|L|L|}
    \hline
        Модель устройства       & Имя устройства                                     & Модули расширения & Общее количество портов (с модулями расширения) & Используемые порты (названия, количество)                            & Свободные порты (названия количество) \\ \hline
        Машрутизатор 4331       & R\allowbreak\_1\allowbreak\_IVANOV                 & -                 & 3 порта GigabitEthernet                         & 3 порта: GigabitEthernet0/0/0-0/0/2                                  & 0 портов \\ \hline
        Машрутизатор 4331       & R\allowbreak\_2\allowbreak\_IVANOV                 & -                 & 3 порта GigabitEthernet                         & 3 порта: GigabitEthernet0/0/0-0/0/2                                  & 0 портов \\ \hline
        Коммутатор L3 3650 24PS & SW\allowbreak\_1\allowbreak\_L3\allowbreak\_IVANOV & -                 & 24 порта GigabitEthernet                        & 7 портов: GigabitEthernet1/0/1 - 1/0/7                               & 14 портов: GigabitEthernet1/0/8 - 1/0/24 \\ \hline
        Коммутатор L3 3650 24PS & SW\allowbreak\_2\allowbreak\_L3\allowbreak\_IVANOV & -                 & 24 порта GigabitEthernet                        & 7 портов: GigabitEthernet1/0/1 - 1/0/7                               & 14 портов: GigabitEthernet1/0/8 - 1/0/24 \\ \hline
        Коммутатор L3 3650 24PS & SW\allowbreak\_1\allowbreak\_L2\allowbreak\_IVANOV & -                 & 24 порта GigabitEthernet                        & 16 портов: GigabitEthernet1/0/1 - 1/0/14                             & 10 портов: GigabitEthernet1/0/15 - 1/0/24 \\ \hline
        Коммутатор 2960         & SW\allowbreak\_2\allowbreak\_L2\allowbreak\_IVANOV & -                 & 24 порта FastEthernet, 2 порта GigabitEthernet  & 20 портов: FastEthernet0/1 - 0/20, 2 порта: GigabitEthernet0/1 - 0/2 & 4 порта: FastEthernet0/21-0/24 \\ \hline
        Коммутатор L3 3650 24PS & SW\allowbreak\_1\allowbreak\_L2\allowbreak\_IVANOV & -                 & 24 порта GigabitEthernet                        & 3 порта: GigabitEthernet1/0/1 - 1/0/3                                & 21 порт: GigabitEthernet1/0/4 - 1/0/24 \\ \hline
  \end{tabulary}
\end{table}
\end{landscape}\pagestyle{plain}\newpage

\begin{figure}[H]
  \centering
  \includegraphics[width=0.95\textwidth]{Topology_FILIAL}
  \caption{Топология сети типового филиала}
  \label{fig:filial_topology}
\end{figure}


\begin{table}[H]
  \caption{План подключений оборудования по портам для типового филиала\label{tab:filial_connection_plan}}
  \centering
  \small
  \begin{tabularx}{\textwidth}{|s|s|s|}
    \hline
    Название устройства                & Порт                         & Описание подключения \\ \hline
    R\_1\_IVANOV      & GigabitEthernet0/0/0         & R\_2\_IVANOV         \\ \cline{2-3}
                                       & GigabitEthernet0/0/1         & SW\_1\_L3\_IVANOV    \\ \cline{2-3}
                                       & GigabitEthernet0/0/2         & SW\_2\_L3\_IVANOV    \\ \hline
    R\_2\_IVANOV      & GigabitEthernet0/0/0         & R\_1\_IVANOV         \\ \cline{2-3}
                                       & GigabitEthernet0/0/1         & SW\_1\_L3\_IVANOV    \\ \cline{2-3}
                                       & GigabitEthernet0/0/2         & SW\_2\_L3\_IVANOV    \\ \hline
    SW\_2\_L3\_IVANOV & GigabitEthernet1/0/1 - 1/0/3 & SW\_1\_L2\_IVANOV    \\ \cline{2-3}
                                       & GigabitEthernet1/0/4         & SW\_2\_L2\_IVANOV    \\ \cline{2-3}
                                       & GigabitEthernet1/0/5         & SW\_3\_L2\_IVANOV    \\ \cline{2-3}
                                       & GigabitEthernet1/0/6         & R\_1\_IVANOV         \\ \cline{2-3}
                                       & GigabitEthernet1/0/7         & R\_2\_IVANOV         \\ \cline{2-3}
                                       & GigabitEthernet1/0/8         & SW\_1\_L3\_IVANOV    \\
  \end{tabularx}
\end{table}

\begin{table}[H]
  \caption*{Продолжение таблицы\;\ref{tab:filial_connection_plan}}
  \centering
  \small
  \begin{tabularx}{\textwidth}{|s|s|s|}
    \hline
    SW\_1\_L2\_IVANOV & GigabitEthernet1/0/1-1/0/10    & PC\_1\_RULE\_IVANOV - PC\_5\_BUKH\_IVANOV \\ \cline{2-3}
                                       & GigabitEthernet1/0/11 - 1/0/12 & SW\_1-2\_L3\_IVANOV                       \\ \hline
    SW\_2\_L2\_IVANOV & FastEthernet0/1-0/20           & PC\_1\_BUY\_IVANOV - PC\_1\_AHO\_IVANOV   \\ \cline{2-3}
                                       & GigabitEthernet0/1 - 0/2       & SW\_1-2\_L3\_IVANOV                       \\ \hline
    SW\_3\_L2\_IVANOV & GigabitEthernet1/0/1           & SERV\_1\_IVANOV                           \\ \cline{2-3}
                                       & GigabitEthernet1/0/2 - 1/0/3   & SW\_1-2\_L3\_IVANOV                       \\ \hline
  \end{tabularx}
\end{table}

Как уже было сказано ранее в рамках типового склада было принято
решение отказа от трехуровневой топологии сети ввиду крайне малого
количества конечных устройств.

Логическая топология типового склада представлена на
Рисунке\;\ref{fig:warehouse_topology}. В данной топологии используется
только одно сетевое устройство --- маршрутизатор, к которому напрямую
подключены конечные устройства площадки. В качестве маршрутизатора
R\_1\_IVANOV используется модель 1841 с одним модулем расширения
HWIC-4ESW, что позволяет подключить 5 портов FastEthernet к конечным
устройствам и иметь один свободный порт FastEthernet.

\begin{figure}[H]
  \centering
  \includegraphics[width=0.5\textwidth]{Topology_WAREHOUSE}
  \caption{Топология сети типового склада}
  \label{fig:warehouse_topology}
\end{figure}

План подключений подключений по портам представлен на Таблице\;\ref{tab:warehouse_connection_plan}.
\begin{table}[H]
  \caption{План подключений оборудования по портам типового склада\label{tab:warehouse_connection_plan}}
  \centering
  \small
  \begin{tabularx}{\textwidth}{|s|s|s|}
    \hline
    Название устройства & Порт            & Описание подключения \\ \hline
    R\_1\_IVANOV        & FastEthernet0-5 & PC\_1\_SEC\_IVANOV - PC\_1\_BRIG\_IVANOV \\ \hline
  \end{tabularx}
\end{table}

\section{ПЛАНИРОВАНИЕ СЕТИ УРОВНЯ 2}
Следующий этап планирования производится на уровне 2 – проектирование
виртуальных локальных сетей. Виртуальные локальные сети можно
разделить на сервисные VLAN, управляющие VLAN и взаимосвязанные VLAN.

При проектировании сервисной виртуальной локальной сети следует
руководствоваться тем, что она предназначена для обеспечения
доступности сервисов для пользователей. Данные VLAN можно назначать на
основе следующих критериев:
\begin{itemize}
\item назначение VLAN по географическому местоположению;
\item назначение VLAN по логической области;
\item назначение VLAN в зависимости от структуры персонала;
\item назначение VLAN по типу услуги.
\end{itemize}

В рамках данной работы оптимальным критерием является назначение в
зависимости от структура персонала. Планирование VLAN для головного
офиса представлено на Таблице\;\ref{tab:hq_vlan_plan}. Типовые филиалы
имеют аналогичную структуру персонала, поэтому планирование VLAN будет
аналогичным, а для типовых складов не имеет смысла в виду отсутсвия
коммутаторов.

\begin{table}[H]
  \caption{Планирование VLAN для головного офиса\label{tab:hq_vlan_plan}}
  \centering
  \small
  \begin{tabularx}{\textwidth}{|s|s|s|}
    \hline
    Идентификатор VLAN & Имя VLAN                & Описание \\ \hline
    101                & ИТ                      & VLAN ИТ-службы \\ \hline
    102                & Sells                   & VLAN Отдела продаж \\ \hline
    103                & Stocks                  & VLAN Отдела закупок \\ \hline
    104                & Accouting               & VLAN Бухгалтерии \\ \hline
    105                & HR                      & VLAN Отдела кадров \\ \hline
    106                & Heads                   & VLAN Руководства \\ \hline
    107                & AHO                     & VLAN АХО \\ \hline
    108                & Security                & VLAN Отдела безопасности \\ \hline
    109                & Server                  & VLAN управления сервером \\ \hline
    110                & Managment\_L2           & VLAN управления устройствами уровня 2 \\ \hline
    111                & Managment\_L3           & VLAN управления устройствами уровня 3 \\
  \end{tabularx}
\end{table}

\begin{table}[H]
  \caption*{Продолжение таблицы\;\ref{tab:hq_vlan_plan}}
  \centering
  \small
  \begin{tabularx}{\textwidth}{|s|s|s|}
    \hline
    112                & Interconnected\_SW1\_R1 & Взаимосвязанная VLAN между SW\_1\_L3\_IVANOV и R\_1\_IVANOV \\ \hline
    113                & Interconnected\_SW2\_R1 & Взаимосвязанная VLAN между SW\_2\_L3\_IVANOV и R\_1\_IVANOV \\ \hline
    114                & Interconnected\_SW1\_R2 & Взаимосвязанная VLAN между SW\_1\_L3\_IVANOV и R\_2\_IVANOV \\ \hline
    115                & Interconnected\_SW2\_R2 & Взаимосвязанная VLAN между SW\_2\_L3\_IVANOV и R\_2\_IVANOV \\ \hline
  \end{tabularx}
\end{table}

Планирование VLAN по портам (Таблица\;\ref{tab:hq_vlan_port_plan})
происходит исходя из логического разделения по отделам, а также
возможности доступа к устройствам уровня 2. Управление всеми сетевыми устройствами
 будет происходить по протоколу SSH\cite{ssh-cisco}.

\begin{table}[H]
  \caption{Планирование VLAN по портам для головного офиса\label{tab:hq_vlan_port_plan}}
  \centering
  \small
  \begin{tabularx}{\textwidth}{|s|s|s|s|s|}
    \hline
    Название устройства                                & Порт                          & Описание подключения                                                                                       & \multicolumn{2}{s|}{VLAN} \\ \cline{4-5}
                                                       &                               &                                                                                                            & Access        & Trunk     \\ \hline
    SW\allowbreak\_1\allowbreak\_L2\allowbreak\_IVANOV & FastEthernet 0/1-0/24         & PC\allowbreak\_1-24\allowbreak\_ИТ\allowbreak\_IVANOV                                                      & 101           &           \\ \cline{2-5}
                                                       & GigabitEthernet 0/1-0/2       & SW\allowbreak\_1-2\allowbreak\_L3\allowbreak\_IVANOV                                                       &               & 101 - 115   \\ \hline
    SW\allowbreak\_2\allowbreak\_L2\allowbreak\_IVANOV & FastEthernet 0/1-0/13         & PC\allowbreak\_1\allowbreak\_BUKH\allowbreak\_IVANOV - PC\allowbreak\_1\allowbreak\_SEC\allowbreak\_IVANOV & 104, 107, 108 &           \\ \cline{2-5}
                                                       & GigabitEthernet 0/1-0/2       & SW\allowbreak\_1-2\allowbreak\_L3\allowbreak\_IVANOV                                                       &               & 101 - 115 \\ \hline
    SW\allowbreak\_3\allowbreak\_L2\allowbreak\_IVANOV & GigabitEthernet 1/0/1-1/0/17  & PC\allowbreak\_1\allowbreak\_HR\allowbreak\_IVANOV - PC\allowbreak\_3\allowbreak\_RULE\allowbreak\_IVANOV  & 101, 105, 106 &           \\ \cline{2-5}
                                                       & GigabitEthernet 1/0/18-1/0/21 & SW\allowbreak\_1-2\allowbreak\_L3\allowbreak\_IVANOV                                                       &               & 101 - 115 \\
  \end{tabularx}
\end{table}

\begin{table}[H]
  \caption*{Продолжение таблицы\;\ref{tab:hq_vlan_port_plan}}
  \centering
  \small
  \begin{tabularx}{\textwidth}{|s|s|s|s|s|}
    \hline
    SW\allowbreak\_4\allowbreak\_L2\allowbreak\_IVANOV & FastEthernet 0/1-0/22         & PC\allowbreak\_1\allowbreak\_SELL\allowbreak\_IVANOV - PC\allowbreak\_11\allowbreak\_BUY\allowbreak\_IVANOV & 102, 103      & \\ \cline{2-5}
                                                                        & GigabitEthernet 0/1-0/2       & SW\allowbreak\_1-2\allowbreak\_L3\allowbreak\_IVANOV                                                        &               & 101 - 115 \\ \hline
    SW\allowbreak\_5\allowbreak\_L2\allowbreak\_IVANOV & GigabitEthernet 1/0/1         & SERV\allowbreak\_1\allowbreak\_IVANOV                                                                       & 109           & \\ \cline{2-5}
                                                                        & GigabitEthernet 1/0/1 - 1/0/2 & SW\allowbreak\_1-2\allowbreak\_L3\allowbreak\_IVANOV                                                        &               & 101 - 115 \\ \hline
    SW\allowbreak\_1\allowbreak\_L3\allowbreak\_IVANOV & GigabitEthernet 1/0/1-1/0/7   & SW\allowbreak\_1-5\allowbreak\_L2\allowbreak\_IVANOV                                                        &               & 101-111 \\ \cline{2-5}
                                                                        & GigabitEthernet 1/0/8         & SW\allowbreak\_2\allowbreak\_L3\allowbreak\_IVANOV                                                          &               & 101 - 115 \\ \cline{2-5}
                                                                        & GigabitEthernet 1/0/9-1/0/10  & R\allowbreak\_1-2\allowbreak\_IVANOV                                                                        &               & 101 - 115 \\ \hline
    SW\allowbreak\_2\allowbreak\_L3\allowbreak\_IVANOV & GigabitEthernet 1/0/1-1/0/7   & SW\allowbreak\_1-5\allowbreak\_L2\allowbreak\_IVANOV                                                        &               & 101-111 \\ \cline{2-5}
                                                                        & GigabitEthernet 1/0/8         & SW\allowbreak\_1\allowbreak\_L3\allowbreak\_IVANOV                                                          &               & 101 - 115 \\ \cline{2-5}
                                                                        & GigabitEthernet 1/0/9-1/0/10  & R\allowbreak\_1-2\allowbreak\_IVANOV                                                                        &               & 101 - 115 \\ \hline
    R\allowbreak\_1\allowbreak\_IVANOV & GigabitEthernet 0/0/1-0/0/2 & SW\allowbreak\_1-2\allowbreak\_L3\allowbreak\_IVANOV &  & \\ \cline{2-5}
                                                        & GigabitEthernet 0/0/0       & R\allowbreak\_2\allowbreak\_IVANOV                   &  & \\ \hline
    R\allowbreak\_2\allowbreak\_IVANOV & GigabitEthernet 0/0/1-0/0/2 & SW\allowbreak\_1-2\allowbreak\_L3\allowbreak\_IVANOV &  & \\ \cline{2-5}
                                                        & GigabitEthernet 0/0/0       & R\allowbreak\_1\allowbreak\_IVANOV                   &  & \\ \hline
  \end{tabularx}
\end{table}

Также для увеличения пропускной способности сети и увеличения
надежности сети путем резервирование будет использоваться
агрегирование каналов Ethernet. Для этой задачи как раз и были
выделены такое большое количество каналов на уровне агрегирования и
ядра (Рисунок\;\ref{fig:hq_topology}). Однако резервный канал можно
создать только между коммутаторами L3 и коммутатору SW\_3\_L2\_IVANOV,
поскольку было использовано максимальное количество портов на
маршрутизаторах (данные каналы будут агрегироваться для увеличения
пропускной способности).

Планирование сетей уровня 2 для типового филиала аналогично
планированию для головного офиса, без выделения VLAN для ИТ-службы,
т.к. её нет в филиалах. Остальные сети VLAN аналогичны основному
зданию, и будут иметь те же идентификаторы, что и в основном здании.

Для типового склада планирование сети VLAN нецелесообразно, поскольку
в топологии сети типовых складов используются только один
маршрутизатор.

Для предотвращения петель уровня 2 будет использоваться протокол
PVST\cite{stp-for-small} в виду его возможности работы с VLAN. Данная
особенность позволяет создавать индивидуальные настройки для каждого
VLAN, а также сделать в будущем балансировку нагрузки (при текущих
требованиях предприятия она не требуется).  Корневым мостом будет
является коммутатор SW\_1\_L3\_IVANOV с доступом ко всем
пользовательским VLAN, в то время как SW\_2\_L3\_IVANOV будет
резервным корневым мостом. Аналогично для филиала и нецелесообразно
для склада.

Также для соблюдения пропускной способности для руководства необходимо
использовать агрегирование каналов. Для этого можно воспользоваться
технологией Etherchannel\cite{cisco-etherchannel}. В качестве
протокола агрегации будет использоваться LACP. Планирование
агрегации каналов для основного здания и филиала представлены в
Таблицах\;\ref{tab:hq_aggregation_plan}~---~\ref{tab:filial_aggregation_plan}.

\begin{table}[H]
  \caption{Планирование агрегации каналов для основного здания\label{tab:hq_aggregation_plan}}
  \centering
  \small
  \begin{tabularx}{\textwidth}{|s|s|s|}
    \hline Устройство в активном режиме & Используемые порты              & Устройство в пассивном режиме работы \\ \hline
    SW\_3\_L2\_IVANOV                   & GigabitEthernet 1/0/18 - 1/0/20 & SW\_1\_L3\_IVANOV                    \\ \cline{2-3}
                                        & GigabitEthernet 1/0/21 - 1/0/23 & SW\_2\_L3\_IVANOV                    \\ \hline
  \end{tabularx}
\end{table}

\begin{table}[H]
  \caption{Планирование агрегации каналов для типового филиала\label{tab:filial_aggregation_plan}}
  \centering
  \small
  \begin{tabularx}{\textwidth}{|s|s|s|}
    \hline Устройство в активном режиме & Используемые порты              & Устройство в пассивном режиме работы \\ \hline
    SW\_1\_L2\_IVANOV                   & GigabitEthernet 1/0/11 - 1/0/12 & SW\_1\_L3\_IVANOV                    \\ \cline{2-3}
                                        & GigabitEthernet 1/0/13 - 1/0/16 & SW\_2\_L3\_IVANOV                    \\ \hline
  \end{tabularx}
\end{table}
\section{ПЛАНИРОВАНИЕ СЕТИ УРОВНЯ 3}

Следующий этап планирования сети --- проектирование распределения
IP-адресов. В качестве диапазона частных адресов был выбран класс C
для каждой из площадок в виду небольшого количества пользователей
(максимальное число адресов хостов --- 256, максимальное количество
конечных устройств --- 77 c учетом сервера). 

Планирование адресации для головного офиса представлено в
Таблице\;\ref{tab:hq_ip_plan}.

\begin{table}[H]
  \caption{Планирование адресации для головного офиса\label{tab:hq_ip_plan}}
  \centering
  \small
  \begin{tabularx}{\textwidth}{|>{\hsize=0.7\hsize}s|>{\hsize=0.6\hsize}s|>{\hsize=1.7\hsize}s|}
    \hline
    Сегмент/маска IP-сети & Адрес шлюза     & Описание сегмента сети \\ \hline
    192.168.101.0/24      & 192.168.101.254 & Сегмент сети, к которому относятся ИТ-служба, со шлюзом, расположенным на коммутаторе уровня агрегации \\ \hline
    192.168.102.0/24      & 192.168.102.254 & Сегмент сети, к которому относятся отдел продаж, со шлюзом, расположенным на коммутаторе уровня агрегации \\ \hline
    192.168.103.0/24      & 192.168.103.254 & Сегмент сети, к которому относятся отдел закупок, со шлюзом, расположенным на коммутаторе уровня агрегации \\ \hline
    192.168.104.0/24      & 192.168.104.254 & Сегмент сети, к которому относятся бухгалтерия, со шлюзом, расположенным на коммутаторе уровня агрегации \\ \hline
    192.168.105.0/24      & 192.168.105.254 & Сегмент сети, к которому относятся отдел кадров, со шлюзом, расположенным на коммутаторе уровня агрегации \\ \hline
    192.168.106.0/24      & 192.168.106.254 & Сегмент сети, к которому относятся руководство, со шлюзом, расположенным на коммутаторе уровня агрегации \\ \hline
    192.168.107.0/24      & 192.168.107.254 & Сегмент сети, к которому относятся АХО, со шлюзом, расположенным на коммутаторе уровня агрегации \\ \hline
    192.168.108.0/24      & 192.168.108.254 & Сегмент сети, к которому относятся отдел безопасности, со шлюзом, расположенным на коммутаторе уровня агрегации \\
  \end{tabularx}
\end{table}

\begin{table}[H]
  \caption*{Продолжение таблицы\;\ref{tab:hq_ip_plan}}
  \centering
  \small
  \begin{tabularx}{\textwidth}{|>{\hsize=0.7\hsize}s|>{\hsize=0.6\hsize}s|>{\hsize=1.7\hsize}s|}
    \hline
    192.168.109.0/24 & 192.168.109.254 & Сегмент сети, к которому относятся сетевые сервисы, со шлюзом, расположенным на коммутаторе уровня агрегации \\ \hline
    192.168.110.0/24 & 192.168.110.254 & Сегмент управляющей сети для устройств уровня 2 со шлюзом, расположенным на коммутаторе уровня агрегации     \\ \hline
    192.168.111.0/24 &                 & Сегмент управляющей сети для устройств уровня 3                                                              \\ \hline
    192.168.112.0/30 &                 & Сегмент сети между SW\_1\_L3\_IVANOV и R\_1\_IVANOV                                                          \\ \hline
    192.168.113.0/30 &                 & Сегмент сети между SW\_2\_L3\_IVANOV и R\_1\_IVANOV                                                          \\ \hline
    192.168.114.0/30 &                 & Сегмент сети между SW\_1\_L3\_IVANOV и R\_2\_IVANOV                                                          \\ \hline
    192.168.115.0/30 &                 & Сегмент сети между SW\_2\_L3\_IVANOV и R\_2\_IVANOV                                                          \\ \hline
    192.168.116.0/30 &                 & Сегмент сети между устройствами уровня ядра                                                                  \\ \hline
    1.1.1.1/32       &                 & Адрес loopback-интерфейса для маршрутизатора R\_1\_IVANOV                                                    \\ \hline
    2.2.2.2/32       &                 & Адрес loopback-интерфейса для маршрутизатора R\_2\_IVANOV                                                    \\ \hline
  \end{tabularx}
\end{table}

Далее необходимо провести планирование режима распределения IP-адресов
(Таблица\;\ref{tab:hq_dhcp_plan}). Для подсетей, которые назначены
конечным устройствам пользователей, назначение будет происходить на
коммутаторе агрегации при помощи DHCP. Для оставшихся подсетей
распределение происходит статическим назначением адресов.

\begin{table}[H]
  \caption{Планирование режима распределения IP-адресов для головного офиса\label{tab:hq_dhcp_plan}}
  \centering
  \small
  \begin{tabularx}{\textwidth}{|s|s|s|}
    \hline
    Сегмент/Интерфейс IP-сети & Режим распределения & Описание режима распределения \\ \hline
    192.168.101.0/24 192.168.102.0/24 192.168.103.0/24 192.168.104.0/24 192.168.105.0/24 192.168.106.0/24 192.168.107.0/24 192.168.108.0/24  & DHCP & Распределяется коммутатором уровня агрегации \\
  \end{tabularx}
\end{table}

\begin{table}[H]
  \caption*{Продолжение таблицы\;\ref{tab:hq_dhcp_plan}}
  \centering
  \small
  \begin{tabularx}{\textwidth}{|s|s|s|}
    \hline
    192.168.109.0/24 & Статический & Статически настроенные IP-адреса сетевых сервисов \\ \hline
    192.168.110.0/24 & Статический & Статически настроенные IP-адреса управления устройствами уровня 2 \\ \hline
    192.168.111.0/24 & Статический & Статически настроенные IP-адреса управления устройствами уровня 3 \\ \hline
    192.168.112.0/30 & Статический & Статически настроенные взаимосвязанные IP-адреса первого коммутатора L3 с первым маршрутизатором \\ \hline
    192.168.113.0/30 & Статический & Статически настроенные взаимосвязанные IP-адреса второго коммутатора L3 с первым маршрутизатором \\ \hline
    192.168.114.0/30 & Статический & Статически настроенные взаимосвязанные IP-адреса первого коммутатора L3 со вторым маршрутизатором \\ \hline
    192.168.115.0/30 & Статический & Статически настроенные взаимосвязанные IP-адреса второго коммутатора L3 со ввторым маршрутизатором \\ \hline
    192.168.116.0/30 & Статический & Статически настроенные взаимосвязанные IP-адреса устройств уровня ядра \\ \hline
  \end{tabularx}
\end{table}

После планирования распределения необходимо провести планирование
маршрутизации (Таблица\;\ref{tab:level_3_summary}). В рамках данной
работы глобальные сети не рассматриваются, поэтому рассматривается
только проектирование внутренних маршрутов. Для устройств
пользователей маршрут по умолчанию устанавливается посредством
DHCP. Для коммутаторов и шлюзов будет использоваться статическая
маршрутизация. Для устройств уровня ядра будет установлен статический
маршрут по умолчанию, который будет перенапрвлять входящий трафик на
loopback-интерфейсы устройств.

\begin{table}[H]
  \caption{Итоги сетевого планирования уровня 3\label{tab:level_3_summary}}
  \centering
  \small
  \begin{tabularx}{\textwidth}{|s|s|s|s|}
    \hline
    IP-сеть       & Метод назначения адреса, шлюз                                              & Режим маршрутизации                            & Описание сети      \\ \hline
    192.168.101.0 & Назначение выполняет SW\_1\_L3\_IVANOV посредством DHCP, SW\_1\_L3\_IVANOV & Анонсирование в OSPF через шлюзовые устройства & Сеть ИТ-службы     \\ \hhline{--~-}
    192.168.102.0 & Назначение выполняет SW\_1\_L3\_IVANOV посредством DHCP, SW\_1\_L3\_IVANOV &                                                         & Сеть отдела продаж \\ \hhline{--~-}
    192.168.103.0 & Назначение выполняет SW\_1\_L3\_IVANOV посредством DHCP, SW\_1\_L3\_IVANOV &                                                         & Сеть отдела закупок \\ \hhline{--~-}
    192.168.104.0 & Назначение выполняет SW\_1\_L3\_IVANOV посредством DHCP, SW\_1\_L3\_IVANOV &                                                         & Сеть бухгалтерии \\ \hhline{--~-}
    192.168.105.0 & Назначение выполняет SW\_1\_L3\_IVANOV посредством DHCP, SW\_1\_L3\_IVANOV &                                                         & Сеть отдела кадров \\ \hhline{--~-}
    192.168.106.0 & Назначение выполняет SW\_1\_L3\_IVANOV посредством DHCP, SW\_1\_L3\_IVANOV &                                                         & Сеть руководства \\ \hhline{--~-}
    192.168.107.0 & Назначение выполняет SW\_1\_L3\_IVANOV посредством DHCP, SW\_1\_L3\_IVANOV &                                                         & Сеть АХО \\
  \end{tabularx}
\end{table}

\begin{table}[H]
  \caption*{Продолжение таблицы\;\ref{tab:level_3_summary}}
  \centering
  \small
  \begin{tabularx}{\textwidth}{|s|s|s|s|}
    \hhline{--~-}
    192.168.108.0 & Назначение выполняет SW\_1\_L3\_IVANOV посредством DHCP, SW\_1\_L3\_IVANOV &                                                         & Сеть отдела безопасности \\ \hhline{--~-}
    192.168.109.0 & Статические адреса, SW\_1\_L3\_IVANOV                                      &                                                         & Сеть сервера \\ \hline
    192.168.110.0 & Статические адреса, SW\_1\_L3\_IVANOV                                      & Маршрут по умолчанию, направленный на SW\_1\_L3\_IVANOV & Сеть управления устройствами уровня 2 \\ \hline
    192.168.111.0 & Статические адреса, шлюз не требуется & Включены OSPF и отношения соседства, маршрутизатор анонсирует маршрут по умолчанию & Сеть управления устройствами уровня 3 \\ \hline
    192.168.112.0 & Статические адреса, шлюз не требуется & Включены OSPF и отношения соседства, маршрутизатор анонсирует маршрут по умолчанию & Сеть для организации связи между SW\_1\_L3\_IVANOV и R\_1\_IVANOV \\ \hline
    192.168.113.0 & Статические адреса, шлюз не требуется & Включены OSPF и отношения соседства, маршрутизатор анонсирует маршрут по умолчанию & Сеть для организации связи между SW\_2\_L3\_IVANOV и R\_1\_IVANOV \\ \hline
    192.168.114.0 & Статические адреса, шлюз не требуется & Включены OSPF и отношения соседства, маршрутизатор анонсирует маршрут по умолчанию & Сеть для организации связи между SW\_1\_L3\_IVANOV и R\_2\_IVANOV \\ \hline
    192.168.115.0 & Статические адреса, шлюз не требуется & Включены OSPF и отношения соседства, маршрутизатор анонсирует маршрут по умолчанию & Сеть для организации связи между SW\_2\_L3\_IVANOV и R\_2\_IVANOV \\
  \end{tabularx}
\end{table}

\begin{table}[H]
  \caption*{Продолжение таблицы\;\ref{tab:level_3_summary}}
  \centering
  \small
  \begin{tabularx}{\textwidth}{|s|s|s|s|}
    \hline
    192.168.116.0 & Статические адреса, шлюз не требуется & Включены OSPF и отношения соседства, маршрутизатор анонсирует маршрут по умолчанию & Сеть для организации связи между маршрутизаторами \\ \hline
  \end{tabularx}
\end{table}

Планирование сетей уровня 3 для типового филиала аналогично
планированию для головного офиса за исключением ИТ-отдела, однако
нумерация остается такой же, как и для основного офиса.

Для типового склада планирование подсетей представлено в
Таблице\;\ref{tab:warehouse_ip_plan}. Маска 24 для бригадира выделена
по причине возможного увеличения количества пользователей в данном
сегменте сети.

\begin{table}[H]
  \caption{Планирование адресации для головного офиса\label{tab:warehouse_ip_plan}}
  \centering
  \small
  \begin{tabularx}{\textwidth}{|>{\hsize=0.7\hsize}s|>{\hsize=0.6\hsize}s|>{\hsize=1.7\hsize}s|}
    \hline
    Сегмент/маска IP-сети & Адрес шлюза     & Описание сегмента сети                                        \\ \hline
    192.168.108.0/24      & 192.168.108.254 & Сегмент сети подключения отдела безопасности к маршрутизатору \\ \hline
    192.168.109.0/24      & 192.168.109.254 & Сегмент сети подключения бригадира к маршрутизатору           \\ \hline
  \end{tabularx}
\end{table}

Для типового склада планирование сетей уровня 3 представлено в
Таблице\;\ref{tab:warehouse_level_3_summary}.

\begin{table}[H]
  \caption{Итоги сетевого планирования уровня 3\label{tab:warehouse_level_3_summary}}
  \centering
  \small
  \begin{tabularx}{\textwidth}{|s|s|s|s|}
    \hline
    IP-сеть       & Метод назначения адреса, шлюз & Режим маршрутизации       & Описание сети                    \\ \hline
    192.168.108.0 & DHCP, R\_1\_IVANOV            & Статическая маршрутизация & Сеть отдела безопасности         \\ \hline
    192.168.109.0 & DHCP, R\_1\_IVANOV            & Статическая маршрутизация & Сеть бригадира складского отдела \\ \hline
  \end{tabularx}
\end{table}

\section{ПЛАНИРОВАНИЕ ПОЛИТИК ФИЛЬТРАЦИИ ТРАФИКА}
В данном пункте работы необходимо провести планирование политик
фильтрации трафика в формате списков контроля доступа.

Схема доступа (Рисунок\;\ref{fig:access_diagram}) данного предприятия
имеет следующую структуру: все отделы имеют доступ в ИТ-службу для
удаленного доступа и администрирования ресурсов, а также все имеют
доступ к сетевым службам, весь остальной доступ ограничен.
\begin{figure}[H]
  \centering
  \includegraphics[width=0.7\textwidth]{access_diagram}
  \caption{Схема доступа}
  \label{fig:access_diagram}
\end{figure}

Списки контроля доступа для большинства отделов будут иметь одинаковый
набор правил, который будет отличаться лишь IP-адресами
источников. Исключением является только список контроля доступа для
ИТ-службы. Все списки контроля доступа, кроме тех, которые относятся к
политикам обеспечения качества обслуживания, будут установлены на
коммутаторы уровня агрегации на виртуальные интерфейсы VLAN как
входные. Пример набора правил для отдела продаж представлен на
Листинге\;\ref{list:acl_sells}. Данный набор правил разрешает отправку
пакетов из отдела продаж для последующего соединения к
маршрутизаторам, а также сетевым службам сервера. Для типовых филиалов
данное планирование аналогично, а для типовых складов нецелесообразно,
поскольку управление доступом будет происходить при помощи статической
маршрутизации сетевого устройства.

\begin{lstlisting}[caption=Команды списков контроля доступа для отдела продаж\label{list:acl_sells}]
ip access-list extended SELLS
 permit ospf any any
 permit udp any any eq bootps
 permit udp host 192.168.102.255 host 192.168.109.1 eq bootpc
 permit ip 192.168.101.0 0.0.0.255 any
 permit ip 192.168.102.0 0.0.0.255 host 192.168.112.2
 permit ip 192.168.102.0 0.0.0.255 host 192.168.113.2
 permit ip 192.168.102.0 0.0.0.255 host 192.168.114.2
 permit ip 192.168.102.0 0.0.0.255 host 192.168.115.2
 permit ip 192.168.102.0 0.0.0.255 host 192.168.116.2
 permit ip 192.168.102.0 0.0.0.255 host 1.1.1.1
 permit ip 192.168.102.0 0.0.0.255 host 2.2.2.2
 permit icmp 192.168.102.0 0.0.0.255 host 192.168.109.1 echo
 permit icmp 192.168.102.0 0.0.0.255 host 192.168.109.1 echo-reply
 permit tcp 192.168.102.0 0.0.0.255 host 192.168.109.1 eq domain
 permit tcp 192.168.102.0 0.0.0.255 host 192.168.109.1 eq www
 permit udp 192.168.102.0 0.0.0.255 host 192.168.109.1 eq domain
 permit tcp 192.168.102.0 0.0.0.255 host 192.168.109.1 eq 389
 permit tcp 192.168.102.0 0.0.0.255 host 192.168.109.1 eq 636
 permit tcp 192.168.102.0 0.0.0.255 host 192.168.109.1 eq 88
 permit tcp 192.168.102.0 0.0.0.255 host 192.168.109.1 eq 464
 permit tcp 192.168.102.0 0.0.0.255 host 192.168.109.1 eq 2049
 permit tcp 192.168.102.0 0.0.0.255 host 192.168.109.1 eq 8385
 permit udp 192.168.102.0 0.0.0.255 host 192.168.109.1 eq 123
 permit udp 192.168.102.0 0.0.0.255 host 192.168.109.1 eq 88
 permit udp 192.168.102.0 0.0.0.255 host 192.168.109.1 eq 464
 permit udp 192.168.102.0 0.0.0.255 host 192.168.109.1 eq 2049
\end{lstlisting}

Пример набора правил для ИТ-службы представлен на
Листинге\;\ref{list:acl_it}. Данная служба должна иметь доступ ко всем
сетевым устройствам для возможности удаленного управления и
исправления ошибок на данных устройствах.
\begin{lstlisting}[caption=Команды списков контроля доступа для ИТ-службы\label{list:acl_it}]
ip access-list extended IT
 permit udp any any eq bootps
 permit udp host 192.168.101.255 host 192.168.109.1 eq bootpc
 permit ip 192.168.101.0 0.0.0.255 any
 permit ospf any any
\end{lstlisting}

При данном планировании используются расширенные списки доступа,
поскольку требуется ограничение по сетевым сервисам. Доступ имеют
только используемые в данной работе сервисы, а также ИТ-служба.

\section{ПЛАНИРОВАНИЕ ПОЛИТИК ОБЕСПЕЧЕНИЯ КАЧЕСТВА ОБСЛУЖИВАНИЯ}
Политики обеспечения качества обслуживания в рамках данной работы
будут реализовываться на основе определения нескольких классов сервиса
и принадлежности трафика определенных сервисов к ним.

Было определено четыре типа класса трафика в рамках работы:
премиальный, золотой, серебряный, бронзовый. Трафик NFS будет
считаться и обрабатываться как премиальный. Трафик веб-службы и службы
управления пользователями как золотой. Серебрянный и бронзовый классы
трафика будут содержать трафик служб динамического конфигурирования
хостов и управления временем соответственно. Все остальное будет
рассматриваться и обрабатываться по модели Best-effort.

Значения DSCP\cite{dscp-value} для классов и типов трафика для
основного здания указаны в Таблице\;\ref{tab:hq_dscp_plan}. Для
типового филиала данные значения будут аналогичны, в то время как для
типовых складов данное планирование неактуально.

\begin{table}[H]
  \caption{Значения DSCP для классов и типов трафика\label{tab:hq_dscp_plan}}
  \centering
  \small
  \begin{tabularx}{\textwidth}{|s|s|s|s|}
    \hline
    Класс трафика & Тип трафика                                  & Модель поведения & Значение DSCP \\ \hline
    Премиальный   & NFS                                          & EF               & 46            \\ \hline
    Золотой       & Служба управления пользователями             & AF11             & 10            \\ \cline{2-4}
                  & Веб-служба                                   & AF21             & 18            \\ \hline
    Серебряный    & Служба динамического конфигурирования хостов & AF31             & 26            \\ \hline
    Бронзовый     & Служба управления временем                   & AF41             & 34            \\ \hline
  \end{tabularx}
\end{table}

Для определения соответствия типов пакетов с политиками обеспечения
качества лучшим способом будет использование списков контроля
доступа. Данные списки контроля доступа представлены в
Приложении\;\ref{apx:qos_acl}.

\section{ПЛАНИРОВАНИЕ СЛУЖБЫ ДОМЕННЫХ ИМЕН}
Планирование службы доменных имен в рамках данной работы будет
производится динамически при помощи сервиса управления пользователями
FreeIPA. Назачение доменов представлено на
Таблице\;\ref{tab:hq_dns_plan}. Как и для предыдущих пунктов работы
планирование службы доменных имен аналогично для типового склада и
неактуально для типового склада.
\begin{table}[H]
  \caption{Планирование доменов\label{tab:hq_dns_plan}}
  \centering
  \small
  \begin{tabularx}{\textwidth}{|s|s|}
    \hline
    Домен & Описание домена \\ \hline
    server.ivanov.test & Домен хоста, а также домен службы управления пользователями \\ \hline
    nfs.ivanov.test & Домен файловой службы \\ \hline
    ntp.ivanov.test & Домен службы времени \\ \hline
    web.ivanov.test & Домен веб-сервера \\ \hline
    rtp.ivanov.test & Домен устройств уровня ядра \\ \hline
    sw.ivanov.test & Домен для коммутаторов \\ \hline
  \end{tabularx}
\end{table}

Домены для коммутаторов и маршрутизаторов требуются для создания ключа
RSA\cite{rsa}, который необходим для конфигурирования протокола
SSH. Данные домены не будут иметь записей в DNS-сервере и будут
использоваться локально в рамках устройства.

Для службы динамического конфигурирования не требуется использование
домена в виду структуры данного сервиса.

Обратная зона DNS сервера будет настроена службой управления
пользователей и будет ``109.168.192.in-addr.arpa''.

\section{ПЛАНИРОВАНИЕ ПОЛИТИК УПРАВЛЕНИЯ ПОЛЬЗОВАТЕЛЯМИ}

В данной работе в качестве службы управления пользователями будет
использоваться FreeIPA. У FreeIPA есть несколько преимуществ, которые
позволили выбрать именно данную службу для использования в данной
курсовой работе:
\begin{enumerate}
\item Данный проект OpenSource, т.е. распротраняется под открытым
  исходным кодом и является бесплатным.
\item Данный проект использует стандартизированный набор сервисов
  (Kerberos в качестве LDAP-сервера, Bind9 в качестве DNS службы).
\item Данный проект разрабатывается гораздо активнее своих аналогов.
\item Данный проект имеет интеграцию с Active Directory --- крайне
  популярным аналогом, который специализируется на ОС Windows.
\end{enumerate}

Ключевым требованием сервера FreeIPA является использование
RPM-основанных дистрибутивов Linux: Fedora, Red Hat Enterprise Linux,
AstraLinux и\;т.д. В рамках данной работы был выбран дистрибутив
Fedora Server 39 для использования в качестве серверной
ОС. Минимальные требования по объему оперативной памяти составляют 1.2
Гб, для 10000 пользователей 4 Гб. Поэтому для реального сервера будет
достаточно 4 Гб оперативной памяти.

В рамках данной работы потребуется использование встроенного DNS
сервера FreeIPA. Также в FreeIPA есть встроенный изолированный сервис
управления временем, однако в данной работе он будет развернут
отдельно. 

Помимо данных сервисов у FreeIPA есть непосредствено сервис
аутентификации пользователей основанный на протоколе Kerberos, и
LDAP-сервер ``389 Directory Server'', которые будут использоваться в
данной работе в рамках планирования политик управления пользователями.

Для более удобного управления и конфигурации у данного программного
обеспечения доступен изолированный веб-сервер Apache с веб-интерфейсом
управления FreeIPA. В нем будут сделаны настройки DNS-сервера, а также
добавлены пользователи.

\section{ПЛАНИРОВАНИЕ ВНЕДРЕНИЯ DHCP-СЕРВЕРА}

При развертывании сервисов будет использоваться программное
обеспечение (ПО) будет использоваться isc-dhcp-server. В Fedora Server
39 пакет называется ``dhcp-server'' версии 4.4.3. В рамках топологии
DHCP сервер будет развернут на сервере SERV\_1\_IVANOV. Благодаря
использованию технологии VLAN, можно будет развернуть сразу несколько
пулов подсетей. Пулы будут развернуты в соответствии с адресацией,
которая описана в Таблице\;\ref{tab:hq_dhcp_plan}.

Параметры для DHCP-сервера представлены в Таблице\;\ref{tab:hq_dhcp_conf}.
\begin{table}[H]
  \caption{Планирование DHCP-сервера\label{tab:hq_dhcp_conf}}
  \centering
  \small
  \begin{tabularx}{\textwidth}{|s|s|}
    \hline
    Параметр & Значение \\ \hline
    Время аренды & 6000 \\ \hline
    Максимальное время аренды & 7200 \\ \hline
    IP-адрес DNS сервера & 192.168.109.1 \\ \hline
  \end{tabularx}
\end{table}

Планирование DHCP-сервера аналогично для типового филиала и склада.

В рамках развертывания сетевых служб для моделирования работы
сервисов, будет развернут DHCP-сервер с сокращенной конфигурацией. В
нем будет присутствовать только подсеть для выдачи адреса клиентской
машине. Данная конфигурация представлена в Приложении\;\ref{apx:virtualbox}.

Более полная конфигурация DHCP-сервера будет сделана уже в программном
обеспечении Cisco Packet Tracer. Конфигурация для данного сервера
представлена в Приложении\;\ref{apx:conf}.

\section{ПЛАНИРОВАНИЕ ДРУГИХ СЕТЕВЫХ СЛУЖБ}

В рамках данной работы также нужно сконфигурировать файловую службу
NFS и службу управлением времени NTP.

Параметры конфигурации NFS-сервера представлены на Таблице\;\ref{tab:hq_nfs_plan}.
\begin{table}[H]
\caption{Планирование NFS-сервера\label{tab:hq_nfs_plan}}
\centering
\small
\begin{tabularx}{\textwidth}{|s|s|}
\hline
/var/nfs/ & /home/<имя клиента>/nfs \\ \hline
Доступ на запись и чтение & Доступ на чтение и запись \\ \hline
Принудительная запись изменений на диск & Принудительная запись изменений на диск \\ \hline
Предотвращение проверки вложенного дерева & Предотвращение проверки вложенного дерева \\ \hline
 & Отсутствие преобразования запросов от root в запросы без привелегий \\ \hline
\end{tabularx}
\end{table}

NFS-сервер будет доступен всем конечным пользователям данной сети. В
качестве программного обеспечения требуется установить пакеты
``nfs-utils'' на сервере и ``nfs-common'' на клиентских машинах.

Служба управления временем NTP будет разворачиваться в рамках ПО
``chrony''. Синхронизация времени будет происходить в соответствии
внутреннем временем на системе сервера.

В качестве веб-службы используется Apache. В отличии от другого
популярного веб-сервера Nginx, система модулей Apache работает при
помощи динамически\cite{apache-vs-nginx} загружаемых модулей. Также
служба Apache использует распределенную\cite{apache-vs-nginx} систему
конфигураций.

Для проверки работоспособности, а также для дальнейшего расчета
нагрузки был создан пример (Листинг\;\ref{list:html}) стартовой
web-страницы компании. Название компании в качестве примера было взято
<<ОптоХим>>.

\begin{lstlisting}[caption=Стартовая страница компании\label{list:html}]
<!DOCTYPE html>
<html>
<head>
    <meta charset="utf-8">
    <title>Оптовая торговля бытовой химией</title>
    <link rel="stylesheet" type="text/css" href="style.css">
</head>
<body>
\end{lstlisting}
\begin{lstlisting}[title=Продолжение листинга\;\ref{list:html}]
    <img src="optohim.png" alt="logo">
    <header>
        <h1>Оптовая торговля бытовой химией</h1>
        <p>Мы заботимся о чистоте вашего дома!</p>
    </header>
    
    <nav>
        <ul>
            <li><a href="#">Каталог продукции</a></li>
            <li><a href="#">О нас</a></li>
            <li><a href="#">Контакты</a></li>
        </ul>
    </nav>
    
    <main>
        <h2>Каталог продукции</h2>

            <div class="product">
        <h3>Средство для мытья посуды "Clean Dish"</h3>
        <p>Описание: Эффективное средство для удаления жира и пятен с посуды. Обладает приятным ароматом и не содержит вредных химических веществ.</p>
        <p>Цена: 200 рублей за 1 литр.</p>
        <p>Количество: Продается в бутылках по 1 литру.</p>
        <p>Дополнительная информация: Подходит для мытья всех типов посуды, включая кастрюли и сковородки. Можно использовать в ручной и машинной мойке.</p>
    </div>
    
    <div class="product">
        <h3>Жидкое мыло "Soft Touch"</h3>
        <p>Описание: Мягкое жидкое мыло с увлажняющими компонентами, которое бережно очищает кожу рук.</p>
        <p>Цена: 150 рублей за 500 мл.</p>
        <p>Количество: Продается во флаконах по 500 мл.</p>
        <p>Дополнительная информация: Содержит натуральные экстракты для увлажнения кожи. Не содержит парабенов и искусственных красителей.</p>
    </div>
    
    <div class="product">
        <h3>Стиральный порошок "Clean Wash"</h3>
        <p>Описание: Высокоэффективный стиральный порошок, который удаляет пятна и освежает ткани.</p>
        <p>Цена: 300 рублей за 1 кг.</p>
        <p>Количество: Продается в упаковках по 1 кг.</p>
        <p>Дополнительная информация: Подходит для стирки белого и цветного белья. Содержит биоразлагаемые компоненты.</p>
    </div>
\end{lstlisting}
\begin{lstlisting}[title=Продолжение листинга\;\ref{list:html}]
        <h2>О нас</h2>
        <p>Наша компания занимается оптовой торговлей бытовой химией уже более 10 лет. Мы предлагаем широкий ассортимент товаров для дома и быта. Наша продукция отличается высоким качеством и доступными ценами.</p>
        
        <h2>Контакты</h2>
        <p>Адрес: <span>Москва, ул. Пушкина, д. 101</span><br>
        Телефон: <span>+7 (999) 123-45-67</span><br>
        E-mail: <span>info@example.com</span></p>
    </main>
    
    <footer>
        <p>© 2022 Оптовая торговля бытовой химией. Все права защищены.</p>
    </footer>
</body>
</html>
\end{lstlisting}

Также был создан примерный логотип (Рисунок\;\ref{fig:company_logo})
компании для создания дополнительной нагрузки к серверу, которая
потребуется в дальнейших расчетах.

\begin{figure}[H]
  \centering
  \includegraphics[width=.5\textwidth]{optohim}
  \caption{Примерный логотип компании}
  \label{fig:company_logo}
\end{figure}

\section{ОПРЕДЕЛЕНИЕ И РАСЧЕТ СЕРВИСНОЙ НАГРУЗКИ}
В ходе данного пункта курсовой работы требуется произвести вычисления
нагрузки сетевых сервисов на корпоративную сеть предприятия. Для этого
будут выполнены вычисления для трех сервисов:
\begin{itemize}
\item веб-служба;
\item файловый сервер;
\item служба управления пользователями.
\end{itemize}

Остальные сервисы будут использоваться во время тестирования нагрузки
данных служб, поэтому они в итоге будут учтены в расчете нагрузки.

Для вычисления нагрузки файловой службы нужно учитывать специфику
предприятия. Поскольку предприятие занимается оптовой продажей, то
нагрузку будут составлять документы в формате\;.pdf.

Для тестирования был воссоздан сеанс работы с файловой службы с
использованием\;.pdf файла размером 1,3 МБ и выполнено скачивание и
загрузка данного файла на файловый сервер. Для анализа данного сеанса
был использовано программное обеспечение Wireshark
(Рисунок\;\ref{fig:wireshark_nfs}). В ходе анализа было определено,
что максимальный размер пакета составляет 65226 бит, количество
пакетов составляет 1007, а время сеанса составило примерно 35 секунд.

\begin{figure}[H]
  \centering
  \includegraphics[width=\textwidth]{wireshark_nfs}
  \caption{Анализ сеанса связи с файловым сервером}
  \label{fig:wireshark_nfs}
\end{figure}

Для данного данных расчетов не будет использоваться средняя длина
пакетов, так как основной объем пакетов идет с данными максимальной
длины. В итоге объем одной транзакции равен
(Формула\;\ref{eq:transaction_volume_nfs}).

\begin{equation}\label{eq:transaction_volume_nfs}
  V_\text{транзакции} = 65226 * 1007 = 65 682 582 \text{бит} \approx 63 \text{Мбит}.
\end{equation}

Далее потребуется определить медианное/пиковое количество
запросов(транзакций) к файловой службе на основе особенностей
предприятия. Количество обращений к файловому серверу происходят от
следующих отделов и в следующем среднем количестве за один рабочий
день (Таблица\;\ref{tab:packet_volume_nfs}).

\begin{table}[H]
  \caption{Планирование количества транзакций к файловой службе\label{tab:packet_volume_nfs}}
  \centering
  \small
  \begin{tabularx}{\textwidth}{|s|s|s|s|}
    \hline
    Отдел         & Количество АРМ & Количество обращений (транзакций) за рабочий день, шт & Всего пакетов за рабочий день, шт \\ \hline
    Отдел ИТ      & 27             & 300                                                   & 302100                            \\ \hline
    Бухгалтерия   & 11             & 700                                                   & 704900                            \\ \hline
    Отдел кадров  & 11             & 670                                                   & 674690                            \\ \hline
    Отдел закупок & 11             & 720                                                   & 725040                            \\ \hline
    Отдел продаж  & 11             & 690                                                   & 694830                            \\ \hline
    АХО           & 1              & 90                                                    & 90630                             \\ \hline
    Дирекция      & 3              & 210                                                   & 211470                            \\ \hline
    Охрана        & 1              & 20                                                    & 20140                             \\ \hline
    Итого         &                & 3400                                                  & 3423800                           \\ \hline
  \end{tabularx}
\end{table}

В итоге за 8-часовой рабочий день формируется примерно 3400 транзакций
для файловой службы, что примерно равно 0,12 транзакций в секунду, что
примерно равно 8 Мбит/с. Это значит что на коммутатор только от
файловой службы будет поступать трафика как минимум на 8 Мбит/с или с
учетом количества пакетов --- 119 пакетов в секунду.

На основе полученных данных требуется с использованием теории очередей
М/М/1\cite{query-theory} оценить минимально необходимую скорость
канала передачи данных.  Вводятся следующие обозначения: $\lambda$ –
скорость прибытия пакетов в секунду, $\mu$ – скорость обслуживания в
пакетах в секунду, утилизация сетевого канала $\rho$,
$V_\text{канала}$ – искомая минимальная скорость канала передачи
данных между коммутатором L3 и сервером в бит/с, $N$ – средний
максимальный объем пакета в битах.

Расчеты искомой минимальной скорости канала передачи данных исходят из
следующих двух формул: расчета коэффициента утилизации сетевого
канала (Формула\;\ref{eq:utilization_koef}) и формулы для расчета
скорости обслуживания (Формула\;\ref{eq:speed_service}). Причем
оптимальное значение коэффициента утилизации канала передачи данных
должно быть меньше $80\%$, так как иначе начнется отбрасывания
пакетов, что не является предпочтительным сценарием.

\begin{equation}\label{eq:utilization_koef}
  \rho = \frac{\lambda}{\mu} \leq 80\% 
\end{equation}
\begin{equation}\label{eq:speed_service}
  \mu = \frac{V_\text{канала}}{N}
\end{equation}

На основе Формул\;\ref{eq:utilization_koef},\;\ref{eq:speed_service}
минимальная скорость канала передачи данных
равна(Формула\;\ref{eq:link_speed}).

\begin{equation}\label{eq:link_speed}
  V_\text{канала} \geq \frac{\lambda  * N}{0,8} 
\end{equation}

В итоге получается, минимальная скорость канала передачи данных равна
9 Мбит/с (Формула\;\ref{eq:link_speed_nfs}).

\begin{equation}\label{eq:link_speed_nfs}
  V_\text{канала} \geq \frac{119 * 65226}{0,8}
\end{equation}

При вычислении нагрузки веб-службы на сеть был организован сеанс
работы с данной службой, то есть открытие веб-страницы из браузера на
адрес web.ivanov.test:8385. После анализа полученных пакетов
(Рисунок\;\ref{fig:wireshark_http}) было определено, что максимальный
размер пакета составляет 1690 бит, количество пакетов за один сеанс
связи составляет 259, время сеанса составило примерно 19 секунд.

\begin{figure}[H]
  \centering
  \includegraphics[width=\textwidth]{wireshark_http}
  \caption{Анализ сеанса связи с веб-службой}
  \label{fig:wireshark_http}
\end{figure}

Таким образом объем одной транзакции равен
(Формула\;\ref{eq:transaction_volume_http}).

\begin{equation}\label{eq:transaction_volume_http}
  V_\text{транзакции} = 1690 * 259 = 437710 \text{бит} \approx 0,4 \text{Мбит}.
\end{equation}

Далее потребуется определить медианное/пиковое количество
запросов(транзакций) к веб-службе на основе особенностей
предприятия. Количество обращений к веб-серверу происходят от
следующих отделов и в следующем среднем количестве за один рабочий
день (Таблица\;\ref{tab:packet_volume_http}).

\begin{table}[H]
  \caption{Планирование количества транзакций к веб-службе\label{tab:packet_volume_http}}
  \centering
  \small
  \begin{tabularx}{\textwidth}{|s|s|s|s|}
    \hline
    Отдел         & Количество АРМ & Количество обращений (транзакций) за рабочий день, шт & Всего пакетов за рабочий день, шт \\ \hline
    Отдел ИТ      & 27             & 3000                                                  & 777000                            \\ \hline
    Бухгалтерия   & 11             & 200                                                   & 51800                             \\ \hline
    Отдел кадров  & 11             & 600                                                   & 155400                            \\ \hline
    Отдел закупок & 11             & 780                                                   & 202020                            \\ \hline
    Отдел продаж  & 11             & 790                                                   & 204610                            \\ \hline
    АХО           & 1              & 80                                                    & 20720                             \\ \hline
    Дирекция      & 3              & 50                                                    & 12950                             \\ \hline
    Охрана        & 1              & 30                                                    & 7770                              \\ \hline
    Итого         &                & 5530                                                  & 1432270                           \\ \hline
  \end{tabularx}
\end{table}

В итоге за 8-часовой рабочий день формируется примерно 5530 транзакций
для файловой службы, что примерно равно 0,19 транзакций в секунду, что
примерно равно 1,4 Мбит/с. Это значит что на коммутатор только от
веб-службы будет поступать трафика как минимум на 1,4 Мбит/с или
с учетом количества пакетов --- 50 пакетов в секунду.

\begin{figure}[H]
  \centering
  \includegraphics[width=\textwidth]{wireshark_http}
  \caption{Анализ сеанса связи с веб-службой}
  \label{fig:wireshark_http}
\end{figure}

На основе Формулы\;\ref{eq:link_speed} в итоге получается, минимальная
скорость канала передачи данных равна 103 Кбит/с
(Формула\;\ref{eq:link_speed_http}).

\begin{equation}\label{eq:link_speed_http}
  V_\text{канала} \geq \frac{50 * 1690}{0,8}
\end{equation}

Для расчета пропускной способности службы управления пользователями
был организован сеанс связи, при котором было произведено логирование
сетевого пользователя и создание билета Kerberos. В ходе анализа
трафика (Рисунок\;\ref{fig:wireshark_ipa}) было определено, что
максимальный объем пакета составляет 2462 бит, количество пакетов 199,
время сеанса связи 72 секунды.

\begin{figure}[H]
  \centering
  \includegraphics[width=\textwidth]{wireshark_ipa}
  \caption{Анализ сеанса связи со службой управления пользователями}
  \label{fig:wireshark_ipa}
\end{figure}

Таким образом объем транзакции составляет
(Формула\;\ref{eq:transaction_volume_ipa}).

\begin{equation}\label{eq:transaction_volume_ipa}
  V_\text{транзакции} = 2462 * 199 = 489938 \text{бит} \approx 0,5 \text{Мбит}.
\end{equation}


Далее потребуется определить медианное/пиковое количество
запросов(транзакций) к службе управления пользователями на основе
особенностей предприятия. Количество обращений к службе управления
пользователями происходят от следующих отделов и в следующем среднем
количестве за один рабочий день
(Таблица\;\ref{tab:packet_volume_ipa}).

\begin{table}[H]
  \caption{Планирование количества транзакций к службе управления пользователями\label{tab:packet_volume_ipa}}
  \centering
  \small
  \begin{tabularx}{\textwidth}{|s|s|s|s|}
    \hline
    Отдел         & Количество АРМ & Количество обращений (транзакций) за рабочий день, шт & Всего пакетов за рабочий день, шт \\ \hline
    Отдел ИТ      & 27             & 2700                                                  & 537300                            \\ \hline
    Бухгалтерия   & 11             & 1100                                                  & 218900                            \\ \hline
    Отдел кадров  & 11             & 2200                                                  & 437800                            \\ \hline
    Отдел закупок & 11             & 3300                                                  & 656700                            \\ \hline
    Отдел продаж  & 11             & 1100                                                  & 218900                            \\ \hline
    АХО           & 1              & 200                                                   & 39800                             \\ \hline
    Дирекция      & 3              & 1500                                                  & 298500                            \\
  \end{tabularx}
\end{table}

\begin{table}[H]
  \caption*{Продолжение таблицы\;\ref{tab:packet_volume_ipa}}
  \centering
  \small
  \begin{tabularx}{\textwidth}{|s|s|s|s|}
    \hline
    Охрана        & 1              & 200                                                   & 39800                             \\ \hline
    Итого         &                & 12300                                                 & 2447700                           \\ \hline
  \end{tabularx}
\end{table}

В итоге за 8-часовой рабочий день формируется примерно 12300
транзакций для файловой службы, что примерно равно 0,43 транзакций в
секунду, что примерно равно 0,215 Мбит/с. Это значит что на коммутатор
только от службы управления пользователями будет поступать трафика как
минимум на 0,215 Мбит/с или с учетом количества пакетов --- 85 пакетов в
секунду.

На основе Формулы\;\ref{eq:link_speed} в итоге получается, минимальная
скорость канала передачи данных равна 255 Кбит/с
(Формула\;\ref{eq:link_speed_ipa}).

\begin{equation}\label{eq:link_speed_ipa}
  V_\text{канала} \geq \frac{85 * 2462}{0,8}
\end{equation}

В итоге можно сделать вывод о том, что канала в 1 Гбит/с хватит для
подддержания пропускной способности и работы сервисов, как было
предположено при предварительном планировании прототипа.  Следует
обратить внимание на то, что в данных расчетах не учитываются
механизмы дифференцированных сервисов и принимается, что очередь одна
и пакеты попадают в нее последовательно.

\section{ИТОГОВОЕ ПЛАНИРОВАНИЕ}

В ходе планирования, которое было выполнено ранее, можно сказать, что
предварительный прототип сети полностью соотвествует требованиям сети
и не требует никаких дополнительных изменений. Однако стоит отметить,
что при использовании текущих линий на уровне ядра пропускной
способности не хватит для соблюдения всех требований сети. При этом в
рамках используемых устройств и слотов расширения нельзя полностью
соблюсти требуемую пропускную способность сети. Для этого необходимо
использовать линии с пропускной способностью 10 Gigabit Ethernet,
которая имеет пропускную способность 10 Гбит/с.

Суммарная стоимость и мощность промежуточных устройств, которые
используются в прототипе сети указаны в
Таблицах\;\ref{tab:hq_price_plan}~---~\ref{tab:warehouse_price_plan}.

\begin{table}[H]
  \caption{Планируемая стоимость и мощность промежуточных устройств для основного здания\label{tab:hq_price_plan}}
  \centering
  \small
  \begin{tabularx}{\textwidth}{|>{\hsize=1.24\hsize}s|s|s|s|s|>{\hsize=.75\hsize}s|}
    \hline
    Название устройства      & Количество устройств, шт & Стоимость одного устройства, у.е. & Мощность одного устройства, Ватт & Суммарная стоимость, у.е. & Суммарная мощность, Ватт \\ \hline
    Маршрутизатор 4331       & 2                        & 3150                              & 535                              & 6300                      & 1070 \\ \hline
    Коммутатор L3 3650 24 PS & 4                        & 5100                              & 54                               & 20400                     & 216 \\ \hline
    Коммутатор L2 2960       & 3                        & 1500                              & 40                               & 4500                      & 120 \\ \hline
    \multicolumn{4}{|s|}{Итого:}  & 31200                    & 1406 \\ \hline
  \end{tabularx}
\end{table}

\begin{table}[H]
  \caption{Планируемая стоимость и мощность промежуточных устройств для типового филиала\label{tab:filial_price_plan}}
  \centering
  \small
  \begin{tabularx}{\textwidth}{|>{\hsize=1.24\hsize}s|s|s|s|s|>{\hsize=.75\hsize}s|}
    \hline
    Название устройства      & Количество устройств, шт & Стоимость одного устройства, у.е. & Мощность одного устройства, Ватт & Суммарная стоимость, у.е. & Суммарная мощность, Ватт \\ \hline
    Маршрутизатор 4331       & 2                        & 3150                              & 535                              & 6300                      & 1070 \\ \hline
    Коммутатор L3 3650 24 PS & 4                        & 5100                              & 54                               & 20400                     & 216 \\ \hline
    Коммутатор L2 2960       & 1                        & 1500                              & 40                               & 1500                      & 40 \\ \hline
    \multicolumn{4}{|s|}{Итого:}  & 28200                    & 1326 \\ \hline
  \end{tabularx}
\end{table}

\begin{table}[H]
  \caption{Планируемая стоимость и мощность промежуточных устройств для основного здания\label{tab:warehouse_price_plan}}
  \centering
  \small
  \begin{tabularx}{\textwidth}{|>{\hsize=1.24\hsize}s|s|s|s|s|>{\hsize=.75\hsize}s|}
    \hline
    Название устройства     & Количество устройств, шт & Стоимость одного устройства, у.е. & Мощность одного устройства, Ватт & Суммарная стоимость, у.е. & Суммарная мощность, Ватт \\ \hline
    Маршрутизатор 1841      & 1                        & 500                               & 35                               & 500                       & 35 \\ \hline
    \multicolumn{4}{|s|}{Итого:} & 500                      & 35 \\ \hline
  \end{tabularx}
\end{table}

С учетом количества площадок итоговая стоимость всех устройтсв будет равна 318700 у.е. 

Результат моделирования сетевых служб представлен в
Приложении\;\ref{apx:virtualbox}. Конфигурация сетевых устройств
представлены в Приложении\;\ref{apx:conf}. Тестирование сети
представлено в Приложении\;\ref{apx:test}.

\clearpage
\section*{ЗАКЛЮЧЕНИЕ}
\addcontentsline{toc}{section}{ЗАКЛЮЧЕНИЕ}

В ходе данной курсовой работы была спроектирована внутренняя сеть
предприятия, которая осущевствляет оптовую торговлю бытовой химией.

Был проведен анализ исходных данных предприятия, в ходе которого были
определены основные требования к пропускной способности сети.

Было сформировано планирование сетей уровня 2 и 3 с последующим
разделением физических сетей на виртуальные подсети, а также с
последующей маршрутизацией сети.

Был сформирован прототип сети с реализацией политик фильтрации трафика
и обеспечения качества обслуживания. В рамках прототипа были внедрены
службы динамического конфигурирования хостов и доменных имен.

Также было проведено планирование развертывания служб управления
временем, управления пользователями и веб-сервера с последующим
расчетом сервисной нагрузки на сеть предприятия.

Было проведено тестирование прототипа сети передачи данных, которое
показало работоспособность данного проекта.

Таким образом, в рамках данной курсовой работы получилось
спроектировать достаточно надежную сеть, которая соблюдает требования,
определнные исходными данными задачи. Такая сеть может быть развернута
в рамках реального предприятия, однако потребует возможных доработок
(например, реализации подключения к глобальной сети, которая не
рассматривалась в рамках данной курсовой работы, а также использования
более современных сетевых устройств, которые не были доступны в рамках
используемого программного обеспечения).

\begingroup \let\itshape\upshape \sloppy
%\raggedright
\printbibliography[title=СПИСОК ИСПОЛЬЗУЕМЫХ ИСТОЧНИКОВ]\addcontentsline{toc}{section}{СПИСОК ИСПОЛЬЗУЕМЫХ ИСТОЧНИКОВ}
\endgroup

\clearpage
\section*{ПРИЛОЖЕНИЯ}
\addcontentsline{toc}{section}{ПРИЛОЖЕНИЯ}


\begin{appendices}

\titleformat{name=\section}[display]
	{\newpage\normalfont\centering}
        {\bfseries Приложение\ \thesection}{0em}{}{}
\renewcommand{\thesection}{\Asbuk{section}}
\addtocontents{toc}{\protect\setcounter{tocdepth}{0}}

\begin{raggedright}
Приложение\;\ref{apx:qos_acl}~---~Списки контроля доступа для обеспечения политик качества обслуживания\\
Приложение\;\ref{apx:virtualbox}~---~Моделирование сетевых служб\\
Приложение\;\ref{apx:conf}~---~Конфигурация сетевых устройств\\
Приложение\;\ref{apx:test}~---~Тестирование работоспособности сети\\
\end{raggedright}

\section{Списки контроля доступа для обеспечения политик качества обслуживания}
\label{apx:qos_acl}

Списки контроля доступа распределы по типам сервисов, на которые будет применяться политики обеспечения качества обслуживания.

Для файловой службы сервера необходима фильтрация портов TCP/UDP 2049 (Листинг\;\ref{list:qos_acl_nfs}).

\begin{lstlisting}[caption=Список контроля доступа для файловой службы\label{list:qos_acl_nfs}]
ip access-list extended NFS
 permit tcp host 192.168.109.1 192.168.104.0 0.0.0.255 eq 2049
 permit udp host 192.168.109.1 192.168.104.0 0.0.0.255 eq 2049
\end{lstlisting}

Для службы управления пользователями необходима фильтрация по портам TCP/UDP (Листинг\;\ref{list:qos_acl_ipa}), которые используются в рамках данной службы а именно:

TCP порты:
\begin{itemize}
\item 80, 443: HTTP/HTTPS;
\item 389, 636: LDAP/LDAPS;
\item 88, 464: kerberos;
\item 53: bind (служба доменных имен).
\end{itemize}

UDP порты:
\begin{itemize}
\item 88, 464: kerberos;
\item 53: bind (служба доменных имен).
\end{itemize}

\begin{lstlisting}[caption=Список контроля доступа для службы управления пользователями\label{list:qos_acl_ipa}]
ip access-list extended IPA
 permit tcp host 192.168.109.1 192.168.96.0 0.0.15.255 eq 389
 permit tcp host 192.168.109.1 192.168.96.0 0.0.15.255 eq 636
 permit tcp host 192.168.109.1 192.168.96.0 0.0.15.255 eq www
 permit tcp host 192.168.109.1 192.168.96.0 0.0.15.255 eq 443
 permit tcp host 192.168.109.1 192.168.96.0 0.0.15.255 eq 88
 permit tcp host 192.168.109.1 192.168.96.0 0.0.15.255 eq 464
 permit tcp host 192.168.109.1 192.168.96.0 0.0.15.255 eq domain
 permit udp host 192.168.109.1 192.168.96.0 0.0.15.255 eq 88
 permit udp host 192.168.109.1 192.168.96.0 0.0.15.255 eq 464
 permit udp host 192.168.109.1 192.168.96.0 0.0.15.255 eq domain
\end{lstlisting}

Для списков контроля доступа к веб-службе сервера используется фильтрация по TCP порту 8385, который был выделен как порт веб-страницы (Листинг\;\ref{list:qos_acl_web}).

\begin{lstlisting}[caption=Список контроля доступа для веб-сервера\label{list:qos_acl_web}]
ip access-list extended WEB
 permit tcp host 192.168.109.1 192.168.96.0 0.0.15.255 eq 8385
\end{lstlisting}

Для служб динамического конфигурирования хостов и управления временем также идет фильтрация по стандартным для данных служб UDP портам (Листинги\;\ref{list:qos_acl_dhcp}-\ref{list:qos_acl_ntp}).

\begin{lstlisting}[caption=Список контроля доступа для службы динамического конфигурирования хостов\label{list:qos_acl_dhcp}]
ip access-list extended DHCP
 permit udp host 192.168.109.1 192.168.96.0 0.0.15.255 eq bootpc
\end{lstlisting}

\begin{lstlisting}[caption=Список контроля доступа для службы управления временем\label{list:qos_acl_ntp}]
ip access-list extended NTP
 permit udp host 192.168.109.1 192.168.96.0 0.0.15.255 eq 123
\end{lstlisting}

Поскольку разграничение доступа происходит при помощи отдельных списков контроля доступа, которые установлены на логические интерфейсы VLAN на L3 коммутаторах, то для политик best-effort мы разрешаем все пакеты из любого источника (Листинг\;\ref{list:qos_acl_server}).

\begin{lstlisting}[caption=Список контроля доступа для службы динамического конфигурирования хостов\label{list:qos_acl_server}]
ip access-list extended SERVER
 permit ip any any
\end{lstlisting}

\section{Моделирование сетевых служб}
\label{apx:virtualbox}

\begin{figure}[H]
  \centering
  \includegraphics[width=\textwidth]{test_hosts}
  \caption{Конфигурация доменов для сервера}
  \label{fig:test_hosts}
\end{figure}

\begin{figure}[H]
  \centering
  \includegraphics[width=\textwidth]{test_dns_config}
  \caption{Конфигурация службы доменных имен}
  \label{fig:test_dns_config}
\end{figure}

\begin{figure}[H]
  \centering
  \includegraphics[width=\textwidth]{test_nslookup}
  \caption{Тестирование с использованием nslookup}
  \label{fig:test_nslookup}
\end{figure}

\begin{figure}[H]
  \centering
  \includegraphics[width=\textwidth]{test_dhcp_server}
  \caption{Конфигурация DHCP-сервера}
  \label{fig:test_dhcp_server}
\end{figure}

\begin{figure}[H]
  \centering
  \includegraphics[width=\textwidth]{test_dhcp_client}
  \caption{Конфигурация DHCP-клиента}
  \label{fig:test_dhcp_client}
\end{figure}

\begin{figure}[H]
  \centering
  \includegraphics[width=\textwidth]{test_dhcp_get}
  \caption{Результат получения DHCP-клиентом конфигураций}
  \label{fig:test_dhcp_get}
\end{figure}

\begin{figure}[H]
  \centering
  \includegraphics[width=\textwidth]{test_http_conf}
  \caption{Конфигурация веб-сервера}
  \label{fig:test_http_conf}
\end{figure}

\begin{figure}[H]
  \centering
  \includegraphics[width=\textwidth]{test_http_client}
  \caption{Веб-страница из веб-бразуера клиента}
  \label{fig:test_http_client}
\end{figure}

\begin{figure}[H]
  \centering
  \includegraphics[width=\textwidth]{test_ntp_service}
  \caption{Статус службы управления временем на клиенте}
  \label{fig:test_ntp_service}
\end{figure}

\begin{figure}[H]
  \centering
  \includegraphics[width=\textwidth]{test_nfs_config}
  \caption{Конфигурация файловой службы на сервере}
  \label{fig:test_nfs_config}
\end{figure}

\begin{figure}[H]
  \centering
  \includegraphics[width=\textwidth]{test_nfs_ls}
  \caption{Содержимое сетевой директории на сервере}
  \label{fig:test_ntp_timedatectl}
\end{figure}

\begin{figure}[H]
  \centering
  \includegraphics[width=\textwidth]{test_freeipa_ipactl}
  \caption{Статус службы управления пользователями на сервере}
  \label{fig:test_freeipa_ipactl}
\end{figure}

\begin{figure}[H]
  \centering
  \includegraphics[width=\textwidth]{test_freeipa_users}
  \caption{Список пользователей}
  \label{fig:test_ntp_timedatectl}
\end{figure}

\begin{figure}[H]
  \centering
  \includegraphics[width=\textwidth]{test_freeipa_adduser}
  \caption{Результат установки FreeIPA на клиентскую машину}
  \label{fig:test_ntp_adduser}
\end{figure}

\begin{figure}[H]
  \centering
  \includegraphics[width=\textwidth]{test_freeipa_user}
  \caption{Добавленный пользователь}
  \label{fig:test_freeipa_user}
\end{figure}

\begin{figure}[H]
  \centering
  \includegraphics{test_freeipa_whoami}
  \caption{Результат попытки входа в систему под добавленным пользователем}
  \label{fig:test_freeipa_whoami}
\end{figure}

\section{Конфигурация сетевых устройств}
\label{apx:conf}

\begin{lstlisting}[caption=Конфигурация устройства R\_1\_IVANOV\label{list:conf_r1}]
!
version 15.4
no service timestamps log datetime msec
no service timestamps debug datetime msec
service password-encryption
!
hostname R_1_IVANOV
!
!
!
enable password 7 0822455D0A16
!
!
!
!
!
!
ip cef
no ipv6 cef
!
!
!
username SSHAdmin privilege 15 password 7 0822455D0A16
!
!
!
!
!
!
!
!
ip ssh authentication-retries 2
ip ssh time-out 60
no ip domain-lookup
ip domain-name rtp.ivanov.test
!
!
spanning-tree mode pvst
!
!
!
!
!
!
interface Loopback0
 ip address 1.1.1.1 255.255.255.255
!
interface GigabitEthernet0/0/0
 ip address 192.168.116.1 255.255.255.252
 duplex auto
 speed auto
!
interface GigabitEthernet0/0/1
 ip address 192.168.112.2 255.255.255.252
 duplex auto
 speed auto
!
interface GigabitEthernet0/0/2
 ip address 192.168.113.2 255.255.255.252
 duplex auto
 speed auto
!
interface Vlan1
 no ip address
 shutdown
!
router ospf 1
 log-adjacency-changes
 network 192.168.112.0 0.0.0.3 area 0
 network 192.168.113.0 0.0.0.3 area 0
 network 192.168.116.0 0.0.0.3 area 0
 default-information originate
!
router rip
!
ip classless
ip route 0.0.0.0 0.0.0.0 Loopback0 
!
ip flow-export version 9
!
!
!
banner motd #
Unauthorized access is prohibited!
#
!
!
!
!
line con 0
 password 7 0822455D0A16
 login
!
line aux 0
!
line vty 0 4
 password 7 0822455D0A16
 login local
 transport input ssh
line vty 5 14
 password 7 0822455D0A16
 login local
 transport input ssh
line vty 15
 login local
 transport input ssh
!
!
!
end

\end{lstlisting}

\begin{lstlisting}[caption=Конфигурация устройства R\_2\_IVANOV\label{list:conf_r2}]
!
version 15.4
no service timestamps log datetime msec
no service timestamps debug datetime msec
service password-encryption
!
hostname R_2_IVANOV
!
!
!
enable password 7 0822455D0A16
!
!
!
!
!
!
ip cef
no ipv6 cef
!
!
!
username SSHAdmin privilege 15 password 7 0822455D0A16
!
!
!
!
!
!
!
!
ip ssh authentication-retries 2
ip ssh time-out 60
no ip domain-lookup
ip domain-name rtp.ivanov.test
!
!
spanning-tree mode pvst
!
!
!
!
!
!
interface Loopback0
 ip address 2.2.2.2 255.255.255.255
!
interface GigabitEthernet0/0/0
 ip address 192.168.116.2 255.255.255.252
 duplex auto
 speed auto
!
interface GigabitEthernet0/0/1
 ip address 192.168.115.2 255.255.255.252
 duplex auto
 speed auto
!
interface GigabitEthernet0/0/2
 ip address 192.168.114.2 255.255.255.252
 duplex auto
 speed auto
!
interface Vlan1
 no ip address
 shutdown
!
router ospf 1
 log-adjacency-changes
 network 192.168.114.0 0.0.0.3 area 0
 network 192.168.115.0 0.0.0.3 area 0
 network 192.168.116.0 0.0.0.3 area 0
 default-information originate
!
router rip
!
ip classless
ip route 0.0.0.0 0.0.0.0 Loopback0 
!
ip flow-export version 9
!
!
!
banner motd #
Unauthorized access is prohibited!
#
!
!
!
!
line con 0
 password 7 0822455D0A16
 login
!
line aux 0
!
line vty 0 4
 password 7 0822455D0A16
 login local
 transport input ssh
line vty 5 14
 password 7 0822455D0A16
 login local
 transport input ssh
line vty 15
 login local
 transport input ssh
!
!
!
end

\end{lstlisting}

\begin{lstlisting}[caption=Конфигурация устройства SW\_1\_L3\_IVANOV\label{list:conf_sw1_l3}]
!
version 16.3.2
no service timestamps log datetime msec
no service timestamps debug datetime msec
service password-encryption
!
hostname SW_1_L3_IVANOV
!
!
enable password 7 0822455D0A16
!
!
!
!
!
!
ip cef
ip routing
!
no ipv6 cef
!
!
!
username SSHAdmin privilege 15 password 7 0822455D0A16
!
!
!
!
!
!
!
!
!
!
ip ssh authentication-retries 2
ip ssh time-out 60
no ip domain-lookup
ip domain-name sw.ivanov.test
!
!
spanning-tree mode pvst
spanning-tree vlan 101-104,110-112,114 priority 24576
spanning-tree vlan 105-109,113,115 priority 28672
!
class-map match-all EF
 match access-group name NFS
class-map match-all platinum
 match ip dscp ef
class-map match-all AF11
 match access-group name IPA
class-map match-all AF21
 match access-group name WEB
class-map match-all gold
 match ip dscp af11
 match ip dscp af21
class-map match-all AF31
 match access-group name DHCP
class-map match-all silver
 match ip dscp af31
class-map match-all AF41
 match access-group name NTP
class-map match-all bronze
 match ip dscp af31
class-map match-all best-effort
 match access-group name SERVER
!
policy-map SERVER
 class platinum
  priority 500
 class gold
  bandwidth percent 35
 class silver
  bandwidth percent 25
 class bronze
  bandwidth percent 15
  shape average 320000
 class best-effort
!
policy-map SETDSCP
 class EF
  set ip dscp ef
 class AF11
  set ip dscp af11
 class AF21
  set ip dscp af21
 class AF31
  set ip dscp af31
 class AF41
  set ip dscp af41
!
!
!
!
!
interface Port-channel1
 switchport trunk allowed vlan 101-115
 switchport mode trunk
!
interface GigabitEthernet1/0/1
 switchport trunk allowed vlan 101-115
 switchport mode trunk
!
interface GigabitEthernet1/0/2
 switchport trunk allowed vlan 101-115
 switchport mode trunk
!
interface GigabitEthernet1/0/3
 switchport trunk allowed vlan 101-115
 switchport mode trunk
 channel-protocol lacp
 channel-group 1 mode passive
!
interface GigabitEthernet1/0/4
 switchport trunk allowed vlan 101-115
 switchport mode trunk
 channel-protocol lacp
 channel-group 1 mode passive
!
interface GigabitEthernet1/0/5
 switchport trunk allowed vlan 101-115
 switchport mode trunk
 channel-protocol lacp
 channel-group 1 mode passive
!
interface GigabitEthernet1/0/6
 switchport trunk allowed vlan 101-112
 switchport mode trunk
!
interface GigabitEthernet1/0/7
 switchport trunk allowed vlan 101-115
 switchport mode trunk
!
interface GigabitEthernet1/0/8
 switchport trunk allowed vlan 101-115
 switchport mode trunk
!
interface GigabitEthernet1/0/9
 switchport access vlan 112
 switchport mode access
!
interface GigabitEthernet1/0/10
 switchport access vlan 114
 switchport mode access
!
interface GigabitEthernet1/0/11
 switchport trunk allowed vlan 101-115
 switchport mode trunk
!
interface GigabitEthernet1/0/12
!
interface GigabitEthernet1/0/13
!
interface GigabitEthernet1/0/14
!
interface GigabitEthernet1/0/15
!
interface GigabitEthernet1/0/16
!
interface GigabitEthernet1/0/17
!
interface GigabitEthernet1/0/18
!
interface GigabitEthernet1/0/19
!
interface GigabitEthernet1/0/20
!
interface GigabitEthernet1/0/21
!
interface GigabitEthernet1/0/22
!
interface GigabitEthernet1/0/23
!
interface GigabitEthernet1/0/24
!
interface GigabitEthernet1/1/1
!
interface GigabitEthernet1/1/2
!
interface GigabitEthernet1/1/3
!
interface GigabitEthernet1/1/4
!
interface Vlan1
 no ip address
 shutdown
!
interface Vlan101
 mac-address 0050.0f51.3701
 ip address 192.168.101.254 255.255.255.0
 ip helper-address 192.168.109.1
 ip access-group IT in
!
interface Vlan102
 mac-address 0050.0f51.3702
 ip address 192.168.102.254 255.255.255.0
 ip helper-address 192.168.109.1
 ip access-group SELLS in
!
interface Vlan103
 mac-address 0050.0f51.3703
 ip address 192.168.103.254 255.255.255.0
 ip helper-address 192.168.109.1
 ip access-group STOCKS in
!
interface Vlan104
 mac-address 0050.0f51.3704
 ip address 192.168.104.254 255.255.255.0
 ip helper-address 192.168.109.1
 ip access-group BUKH in
!
interface Vlan105
 mac-address 0050.0f51.3705
 ip address 192.168.105.254 255.255.255.0
 ip helper-address 192.168.109.1
 ip access-group HR in
!
interface Vlan106
 mac-address 0050.0f51.3706
 ip address 192.168.106.254 255.255.255.0
 ip helper-address 192.168.109.1
 ip access-group HEADS in
!
interface Vlan107
 mac-address 0050.0f51.3707
 ip address 192.168.107.254 255.255.255.0
 ip helper-address 192.168.109.1
 ip access-group AHO in
!
interface Vlan108
 mac-address 0050.0f51.3708
 ip address 192.168.108.254 255.255.255.0
 ip helper-address 192.168.109.1
 ip access-group SEC in
!
interface Vlan109
 mac-address 0050.0f51.3709
 ip address 192.168.109.254 255.255.255.0
!
interface Vlan110
 mac-address 0050.0f51.370a
 ip address 192.168.110.254 255.255.255.0
!
interface Vlan111
 mac-address 0050.0f51.370b
 ip address 192.168.111.1 255.255.255.0
!
interface Vlan112
 mac-address 0050.0f51.370c
 ip address 192.168.112.1 255.255.255.252
!
interface Vlan114
 mac-address 0050.0f51.370d
 ip address 192.168.114.1 255.255.255.252
!
router ospf 1
 log-adjacency-changes
 network 192.168.101.0 0.0.0.255 area 0
 network 192.168.102.0 0.0.0.255 area 0
 network 192.168.103.0 0.0.0.255 area 0
 network 192.168.104.0 0.0.0.255 area 0
 network 192.168.105.0 0.0.0.255 area 0
 network 192.168.106.0 0.0.0.255 area 0
 network 192.168.107.0 0.0.0.255 area 0
 network 192.168.108.0 0.0.0.255 area 0
 network 192.168.109.0 0.0.0.255 area 0
 network 192.168.110.0 0.0.0.255 area 0
 network 192.168.112.0 0.0.0.3 area 0
 network 192.168.114.0 0.0.0.3 area 0
!
ip classless
!
ip flow-export version 9
!
!
ip access-list extended SERVER
 permit ip any any
ip access-list extended NFS
 permit tcp host 192.168.109.1 192.168.104.0 0.0.0.255 eq 2049
 permit udp host 192.168.109.1 192.168.104.0 0.0.0.255 eq 2049
ip access-list extended IPA
 permit tcp host 192.168.109.1 192.168.96.0 0.0.15.255 eq 389
 permit tcp host 192.168.109.1 192.168.96.0 0.0.15.255 eq 636
 permit tcp host 192.168.109.1 192.168.96.0 0.0.15.255 eq www
 permit tcp host 192.168.109.1 192.168.96.0 0.0.15.255 eq 443
 permit tcp host 192.168.109.1 192.168.96.0 0.0.15.255 eq 88
 permit tcp host 192.168.109.1 192.168.96.0 0.0.15.255 eq 464
 permit tcp host 192.168.109.1 192.168.96.0 0.0.15.255 eq domain
 permit udp host 192.168.109.1 192.168.96.0 0.0.15.255 eq 88
 permit udp host 192.168.109.1 192.168.96.0 0.0.15.255 eq 464
 permit udp host 192.168.109.1 192.168.96.0 0.0.15.255 eq domain
ip access-list extended WEB
 permit tcp host 192.168.109.1 192.168.96.0 0.0.15.255 eq 8385
ip access-list extended DHCP
 permit udp host 192.168.109.1 192.168.96.0 0.0.15.255 eq bootpc
 permit udp any any eq bootps
ip access-list extended NTP
 permit udp host 192.168.109.1 192.168.96.0 0.0.15.255 eq 123
ip access-list extended SELLS
 permit ospf any any
 permit udp any any eq bootps
 permit udp host 192.168.102.255 host 192.168.109.1 eq bootpc
 permit ip 192.168.101.0 0.0.0.255 any
 permit ip 192.168.102.0 0.0.0.255 host 192.168.112.2
 permit ip 192.168.102.0 0.0.0.255 host 192.168.113.2
 permit ip 192.168.102.0 0.0.0.255 host 192.168.114.2
 permit ip 192.168.102.0 0.0.0.255 host 192.168.115.2
 permit ip 192.168.102.0 0.0.0.255 host 192.168.116.2
 permit ip 192.168.102.0 0.0.0.255 host 1.1.1.1
 permit ip 192.168.102.0 0.0.0.255 host 2.2.2.2
 permit icmp 192.168.102.0 0.0.0.255 host 192.168.109.1 echo
 permit icmp 192.168.102.0 0.0.0.255 host 192.168.109.1 echo-reply
 permit tcp 192.168.102.0 0.0.0.255 host 192.168.109.1 eq domain
 permit tcp 192.168.102.0 0.0.0.255 host 192.168.109.1 eq www
 permit udp 192.168.102.0 0.0.0.255 host 192.168.109.1 eq domain
 permit tcp 192.168.102.0 0.0.0.255 host 192.168.109.1 eq 389
 permit tcp 192.168.102.0 0.0.0.255 host 192.168.109.1 eq 636
 permit tcp 192.168.102.0 0.0.0.255 host 192.168.109.1 eq 88
 permit tcp 192.168.102.0 0.0.0.255 host 192.168.109.1 eq 464
 permit tcp 192.168.102.0 0.0.0.255 host 192.168.109.1 eq 2049
 permit tcp 192.168.102.0 0.0.0.255 host 192.168.109.1 eq 8385
 permit udp 192.168.102.0 0.0.0.255 host 192.168.109.1 eq 123
 permit udp 192.168.102.0 0.0.0.255 host 192.168.109.1 eq 88
 permit udp 192.168.102.0 0.0.0.255 host 192.168.109.1 eq 464
 permit udp 192.168.102.0 0.0.0.255 host 192.168.109.1 eq 2049
ip access-list extended STOCKS
 permit ospf any any
 permit udp any any eq bootps
 permit udp host 192.168.103.255 host 192.168.109.1 eq bootpc
 permit ip 192.168.101.0 0.0.0.255 any
 permit ip 192.168.103.0 0.0.0.255 host 192.168.112.2
 permit ip 192.168.103.0 0.0.0.255 host 192.168.113.2
 permit ip 192.168.103.0 0.0.0.255 host 192.168.114.2
 permit ip 192.168.103.0 0.0.0.255 host 192.168.115.2
 permit ip 192.168.103.0 0.0.0.255 host 192.168.116.2
 permit ip 192.168.103.0 0.0.0.255 host 1.1.1.1
 permit ip 192.168.103.0 0.0.0.255 host 2.2.2.2
 permit icmp 192.168.103.0 0.0.0.255 host 192.168.109.1 echo
 permit icmp 192.168.103.0 0.0.0.255 host 192.168.109.1 echo-reply
 permit tcp 192.168.103.0 0.0.0.255 host 192.168.109.1 eq domain
 permit tcp 192.168.103.0 0.0.0.255 host 192.168.109.1 eq www
 permit udp 192.168.103.0 0.0.0.255 host 192.168.109.1 eq domain
 permit tcp 192.168.103.0 0.0.0.255 host 192.168.109.1 eq 389
 permit tcp 192.168.103.0 0.0.0.255 host 192.168.109.1 eq 636
 permit tcp 192.168.103.0 0.0.0.255 host 192.168.109.1 eq 88
 permit tcp 192.168.103.0 0.0.0.255 host 192.168.109.1 eq 464
 permit tcp 192.168.103.0 0.0.0.255 host 192.168.109.1 eq 2049
 permit tcp 192.168.103.0 0.0.0.255 host 192.168.109.1 eq 8385
 permit udp 192.168.103.0 0.0.0.255 host 192.168.109.1 eq 123
 permit udp 192.168.103.0 0.0.0.255 host 192.168.109.1 eq 88
 permit udp 192.168.103.0 0.0.0.255 host 192.168.109.1 eq 464
 permit udp 192.168.103.0 0.0.0.255 host 192.168.109.1 eq 2049
ip access-list extended BUKH
 permit ospf any any
 permit udp any any eq bootps
 permit udp host 192.168.104.255 host 192.168.109.1 eq bootpc
 permit ip 192.168.101.0 0.0.0.255 any
 permit ip 192.168.104.0 0.0.0.255 host 192.168.112.2
 permit ip 192.168.104.0 0.0.0.255 host 192.168.113.2
 permit ip 192.168.104.0 0.0.0.255 host 192.168.114.2
 permit ip 192.168.104.0 0.0.0.255 host 192.168.115.2
 permit ip 192.168.104.0 0.0.0.255 host 192.168.116.2
 permit ip 192.168.104.0 0.0.0.255 host 1.1.1.1
 permit ip 192.168.104.0 0.0.0.255 host 2.2.2.2
 permit icmp 192.168.104.0 0.0.0.255 host 192.168.109.1 echo
 permit icmp 192.168.104.0 0.0.0.255 host 192.168.109.1 echo-reply
 permit tcp 192.168.104.0 0.0.0.255 host 192.168.109.1 eq domain
 permit tcp 192.168.104.0 0.0.0.255 host 192.168.109.1 eq www
 permit udp 192.168.104.0 0.0.0.255 host 192.168.109.1 eq domain
 permit tcp 192.168.104.0 0.0.0.255 host 192.168.109.1 eq 389
 permit tcp 192.168.104.0 0.0.0.255 host 192.168.109.1 eq 636
 permit tcp 192.168.104.0 0.0.0.255 host 192.168.109.1 eq 88
 permit tcp 192.168.104.0 0.0.0.255 host 192.168.109.1 eq 464
 permit tcp 192.168.104.0 0.0.0.255 host 192.168.109.1 eq 2049
 permit tcp 192.168.104.0 0.0.0.255 host 192.168.109.1 eq 8385
 permit udp 192.168.104.0 0.0.0.255 host 192.168.109.1 eq 123
 permit udp 192.168.104.0 0.0.0.255 host 192.168.109.1 eq 88
 permit udp 192.168.104.0 0.0.0.255 host 192.168.109.1 eq 464
 permit udp 192.168.104.0 0.0.0.255 host 192.168.109.1 eq 2049
ip access-list extended HR
 permit ospf any any
 permit udp any any eq bootps
 permit udp host 192.168.105.255 host 192.168.109.1 eq bootpc
 permit ip 192.168.101.0 0.0.0.255 any
 permit ip 192.168.105.0 0.0.0.255 host 192.168.112.2
 permit ip 192.168.105.0 0.0.0.255 host 192.168.113.2
 permit ip 192.168.105.0 0.0.0.255 host 192.168.114.2
 permit ip 192.168.105.0 0.0.0.255 host 192.168.115.2
 permit ip 192.168.105.0 0.0.0.255 host 192.168.116.2
 permit ip 192.168.105.0 0.0.0.255 host 1.1.1.1
 permit ip 192.168.105.0 0.0.0.255 host 2.2.2.2
 permit icmp 192.168.105.0 0.0.0.255 host 192.168.109.1 echo
 permit icmp 192.168.105.0 0.0.0.255 host 192.168.109.1 echo-reply
 permit tcp 192.168.105.0 0.0.0.255 host 192.168.109.1 eq domain
 permit tcp 192.168.105.0 0.0.0.255 host 192.168.109.1 eq www
 permit udp 192.168.105.0 0.0.0.255 host 192.168.109.1 eq domain
 permit tcp 192.168.105.0 0.0.0.255 host 192.168.109.1 eq 389
 permit tcp 192.168.105.0 0.0.0.255 host 192.168.109.1 eq 636
 permit tcp 192.168.105.0 0.0.0.255 host 192.168.109.1 eq 88
 permit tcp 192.168.105.0 0.0.0.255 host 192.168.109.1 eq 464
 permit tcp 192.168.105.0 0.0.0.255 host 192.168.109.1 eq 2049
 permit tcp 192.168.105.0 0.0.0.255 host 192.168.109.1 eq 8385
 permit udp 192.168.105.0 0.0.0.255 host 192.168.109.1 eq 123
 permit udp 192.168.105.0 0.0.0.255 host 192.168.109.1 eq 88
 permit udp 192.168.105.0 0.0.0.255 host 192.168.109.1 eq 464
 permit udp 192.168.105.0 0.0.0.255 host 192.168.109.1 eq 2049
ip access-list extended HEADS
 permit ospf any any
 permit udp any any eq bootps
 permit udp host 192.168.106.255 host 192.168.109.1 eq bootpc
 permit ip 192.168.101.0 0.0.0.255 any
 permit ip 192.168.106.0 0.0.0.255 host 192.168.112.2
 permit ip 192.168.106.0 0.0.0.255 host 192.168.113.2
 permit ip 192.168.106.0 0.0.0.255 host 192.168.114.2
 permit ip 192.168.106.0 0.0.0.255 host 192.168.115.2
 permit ip 192.168.106.0 0.0.0.255 host 192.168.116.2
 permit ip 192.168.106.0 0.0.0.255 host 1.1.1.1
 permit ip 192.168.106.0 0.0.0.255 host 2.2.2.2
 permit icmp 192.168.106.0 0.0.0.255 host 192.168.109.1 echo
 permit icmp 192.168.106.0 0.0.0.255 host 192.168.109.1 echo-reply
 permit tcp 192.168.106.0 0.0.0.255 host 192.168.109.1 eq domain
 permit tcp 192.168.106.0 0.0.0.255 host 192.168.109.1 eq www
 permit udp 192.168.106.0 0.0.0.255 host 192.168.109.1 eq domain
 permit tcp 192.168.106.0 0.0.0.255 host 192.168.109.1 eq 389
 permit tcp 192.168.106.0 0.0.0.255 host 192.168.109.1 eq 636
 permit tcp 192.168.106.0 0.0.0.255 host 192.168.109.1 eq 88
 permit tcp 192.168.106.0 0.0.0.255 host 192.168.109.1 eq 464
 permit tcp 192.168.106.0 0.0.0.255 host 192.168.109.1 eq 2049
 permit tcp 192.168.106.0 0.0.0.255 host 192.168.109.1 eq 8385
 permit udp 192.168.106.0 0.0.0.255 host 192.168.109.1 eq 123
 permit udp 192.168.106.0 0.0.0.255 host 192.168.109.1 eq 88
 permit udp 192.168.106.0 0.0.0.255 host 192.168.109.1 eq 464
 permit udp 192.168.106.0 0.0.0.255 host 192.168.109.1 eq 2049
ip access-list extended AHO
 permit ospf any any
 permit udp any any eq bootps
 permit udp host 192.168.107.255 host 192.168.109.1 eq bootpc
 permit ip 192.168.101.0 0.0.0.255 any
 permit ip 192.168.107.0 0.0.0.255 host 192.168.112.2
 permit ip 192.168.107.0 0.0.0.255 host 192.168.113.2
 permit ip 192.168.107.0 0.0.0.255 host 192.168.114.2
 permit ip 192.168.107.0 0.0.0.255 host 192.168.115.2
 permit ip 192.168.107.0 0.0.0.255 host 192.168.116.2
 permit ip 192.168.107.0 0.0.0.255 host 1.1.1.1
 permit ip 192.168.107.0 0.0.0.255 host 2.2.2.2
 permit icmp 192.168.107.0 0.0.0.255 host 192.168.109.1 echo
 permit icmp 192.168.107.0 0.0.0.255 host 192.168.109.1 echo-reply
 permit tcp 192.168.107.0 0.0.0.255 host 192.168.109.1 eq domain
 permit tcp 192.168.107.0 0.0.0.255 host 192.168.109.1 eq www
 permit udp 192.168.107.0 0.0.0.255 host 192.168.109.1 eq domain
 permit tcp 192.168.107.0 0.0.0.255 host 192.168.109.1 eq 389
 permit tcp 192.168.107.0 0.0.0.255 host 192.168.109.1 eq 636
 permit tcp 192.168.107.0 0.0.0.255 host 192.168.109.1 eq 88
 permit tcp 192.168.107.0 0.0.0.255 host 192.168.109.1 eq 464
 permit tcp 192.168.107.0 0.0.0.255 host 192.168.109.1 eq 2049
 permit tcp 192.168.107.0 0.0.0.255 host 192.168.109.1 eq 8385
 permit udp 192.168.107.0 0.0.0.255 host 192.168.109.1 eq 123
 permit udp 192.168.107.0 0.0.0.255 host 192.168.109.1 eq 88
 permit udp 192.168.107.0 0.0.0.255 host 192.168.109.1 eq 464
 permit udp 192.168.107.0 0.0.0.255 host 192.168.109.1 eq 2049
ip access-list extended SEC
 permit ospf any any
 permit udp any any eq bootps
 permit udp host 192.168.108.255 host 192.168.109.1 eq bootpc
 permit ip 192.168.101.0 0.0.0.255 any
 permit ip 192.168.108.0 0.0.0.255 host 192.168.112.2
 permit ip 192.168.108.0 0.0.0.255 host 192.168.113.2
 permit ip 192.168.108.0 0.0.0.255 host 192.168.114.2
 permit ip 192.168.108.0 0.0.0.255 host 192.168.115.2
 permit ip 192.168.108.0 0.0.0.255 host 192.168.116.2
 permit ip 192.168.108.0 0.0.0.255 host 1.1.1.1
 permit ip 192.168.108.0 0.0.0.255 host 2.2.2.2
 permit icmp 192.168.108.0 0.0.0.255 host 192.168.109.1 echo
 permit icmp 192.168.108.0 0.0.0.255 host 192.168.109.1 echo-reply
 permit tcp 192.168.108.0 0.0.0.255 host 192.168.109.1 eq domain
 permit tcp 192.168.108.0 0.0.0.255 host 192.168.109.1 eq www
 permit udp 192.168.108.0 0.0.0.255 host 192.168.109.1 eq domain
 permit tcp 192.168.108.0 0.0.0.255 host 192.168.109.1 eq 389
 permit tcp 192.168.108.0 0.0.0.255 host 192.168.109.1 eq 636
 permit tcp 192.168.108.0 0.0.0.255 host 192.168.109.1 eq 88
 permit tcp 192.168.108.0 0.0.0.255 host 192.168.109.1 eq 464
 permit tcp 192.168.108.0 0.0.0.255 host 192.168.109.1 eq 2049
 permit tcp 192.168.108.0 0.0.0.255 host 192.168.109.1 eq 8385
 permit udp 192.168.108.0 0.0.0.255 host 192.168.109.1 eq 123
 permit udp 192.168.108.0 0.0.0.255 host 192.168.109.1 eq 88
 permit udp 192.168.108.0 0.0.0.255 host 192.168.109.1 eq 464
 permit udp 192.168.108.0 0.0.0.255 host 192.168.109.1 eq 2049
ip access-list extended IT
 permit udp any any eq bootps
 permit udp host 192.168.101.255 host 192.168.109.1 eq bootpc
 permit ip 192.168.101.0 0.0.0.255 any
 permit ospf any any
!
banner motd #
Unauthorized access is prohibited
#
!
!
!
!
line con 0
 password 7 0822455D0A16
 login
!
line aux 0
!
line vty 0 4
 password 7 0822455D0A16
 login local
 transport input ssh
line vty 5 14
 password 7 0822455D0A16
 login local
 transport input ssh
line vty 15
 login local
 transport input ssh
!
!
!
!
end

\end{lstlisting}

\begin{lstlisting}[caption=Конфигурация устройства SW\_2\_L3\_IVANOV\label{list:conf_sw2_l3}]
!
version 16.3.2
no service timestamps log datetime msec
no service timestamps debug datetime msec
service password-encryption
!
hostname SW_2_L3_IVANOV
!
!
enable password 7 0822455D0A16
!
!
!
!
!
!
ip cef
ip routing
!
no ipv6 cef
!
!
!
username SSHAdmin privilege 15 password 7 0822455D0A16
!
!
!
!
!
!
!
!
!
!
ip ssh authentication-retries 2
ip ssh time-out 60
no ip domain-lookup
ip domain-name sw.ivanov.test
!
!
spanning-tree mode pvst
spanning-tree vlan 105-109,113,115 priority 24576
spanning-tree vlan 101-104,110-112,114 priority 28672
!
class-map match-all EF
 match access-group name NFS
class-map match-all platinum
 match ip dscp ef
class-map match-all AF11
 match access-group name IPA
class-map match-all AF21
 match access-group name WEB
class-map match-all gold
 match ip dscp af11
 match ip dscp af21
class-map match-all AF31
 match access-group name DHCP
class-map match-all silver
 match ip dscp af31
class-map match-all AF41
 match access-group name NTP
class-map match-all bronze
 match ip dscp af31
class-map match-all best-effort
 match access-group name SERVER
!
policy-map SERVER
 class platinum
  priority 500
 class gold
  bandwidth percent 35
 class silver
  bandwidth percent 25
 class bronze
  bandwidth percent 15
  shape average 320000
 class best-effort
  shape average 56000
!
policy-map SETDSCP
 class EF
  set ip dscp ef
 class AF11
  set ip dscp af11
 class AF21
  set ip dscp af21
 class AF31
  set ip dscp af31
 class AF41
  set ip dscp af41
!
!
!
!
!
interface Port-channel1
 switchport trunk allowed vlan 101-115
 switchport mode trunk
!
interface GigabitEthernet1/0/1
 switchport trunk allowed vlan 101-115
 switchport mode trunk
!
interface GigabitEthernet1/0/2
 switchport trunk allowed vlan 101-115
 switchport mode trunk
!
interface GigabitEthernet1/0/3
 switchport trunk allowed vlan 101-115
 switchport mode trunk
 channel-protocol lacp
 channel-group 1 mode passive
!
interface GigabitEthernet1/0/4
 switchport trunk allowed vlan 101-115
 switchport mode trunk
 channel-protocol lacp
 channel-group 1 mode passive
!
interface GigabitEthernet1/0/5
 switchport trunk allowed vlan 101-115
 switchport mode trunk
 channel-protocol lacp
 channel-group 1 mode passive
!
interface GigabitEthernet1/0/6
 switchport trunk allowed vlan 101-113
 switchport mode trunk
!
interface GigabitEthernet1/0/7
 switchport trunk allowed vlan 101-115
 switchport mode trunk
!
interface GigabitEthernet1/0/8
 switchport trunk allowed vlan 101-115
 switchport mode trunk
!
interface GigabitEthernet1/0/9
 switchport access vlan 115
 switchport mode access
!
interface GigabitEthernet1/0/10
 switchport access vlan 113
 switchport mode access
!
interface GigabitEthernet1/0/11
 switchport trunk allowed vlan 101-115
 switchport mode trunk
!
interface GigabitEthernet1/0/12
!
interface GigabitEthernet1/0/13
!
interface GigabitEthernet1/0/14
!
interface GigabitEthernet1/0/15
!
interface GigabitEthernet1/0/16
!
interface GigabitEthernet1/0/17
!
interface GigabitEthernet1/0/18
!
interface GigabitEthernet1/0/19
!
interface GigabitEthernet1/0/20
!
interface GigabitEthernet1/0/21
!
interface GigabitEthernet1/0/22
!
interface GigabitEthernet1/0/23
!
interface GigabitEthernet1/0/24
!
interface GigabitEthernet1/1/1
!
interface GigabitEthernet1/1/2
!
interface GigabitEthernet1/1/3
!
interface GigabitEthernet1/1/4
!
interface Vlan1
 no ip address
 shutdown
!
interface Vlan101
 mac-address 00e0.b0ce.6501
 ip address 192.168.101.254 255.255.255.0
 ip helper-address 192.168.109.1
 ip access-group IT in
!
interface Vlan102
 mac-address 00e0.b0ce.6502
 ip address 192.168.102.254 255.255.255.0
 ip helper-address 192.168.109.1
 ip access-group SELLS in
!
interface Vlan103
 mac-address 00e0.b0ce.6503
 ip address 192.168.103.254 255.255.255.0
 ip helper-address 192.168.109.1
 ip access-group STOCKS in
!
interface Vlan104
 mac-address 00e0.b0ce.6504
 ip address 192.168.104.254 255.255.255.0
 ip helper-address 192.168.109.1
 ip access-group BUKH in
!
interface Vlan105
 mac-address 00e0.b0ce.6505
 ip address 192.168.105.254 255.255.255.0
 ip helper-address 192.168.109.1
 ip access-group HR in
!
interface Vlan106
 mac-address 00e0.b0ce.6506
 ip address 192.168.106.254 255.255.255.0
 ip helper-address 192.168.109.1
 ip access-group HEADS in
!
interface Vlan107
 mac-address 00e0.b0ce.6507
 ip address 192.168.107.254 255.255.255.0
 ip helper-address 192.168.109.1
 ip access-group AHO in
!
interface Vlan108
 mac-address 00e0.b0ce.6508
 ip address 192.168.108.254 255.255.255.0
 ip helper-address 192.168.109.1
 ip access-group SEC in
!
interface Vlan109
 mac-address 00e0.b0ce.6509
 ip address 192.168.109.254 255.255.255.0
!
interface Vlan110
 mac-address 00e0.b0ce.650a
 ip address 192.168.110.254 255.255.255.0
!
interface Vlan111
 mac-address 00e0.b0ce.650b
 ip address 192.168.111.2 255.255.255.0
!
interface Vlan113
 mac-address 00e0.b0ce.650c
 ip address 192.168.113.1 255.255.255.252
!
interface Vlan115
 mac-address 00e0.b0ce.650d
 ip address 192.168.115.1 255.255.255.252
!
router ospf 1
 log-adjacency-changes
 network 192.168.101.0 0.0.0.255 area 0
 network 192.168.102.0 0.0.0.255 area 0
 network 192.168.103.0 0.0.0.255 area 0
 network 192.168.104.0 0.0.0.255 area 0
 network 192.168.105.0 0.0.0.255 area 0
 network 192.168.106.0 0.0.0.255 area 0
 network 192.168.107.0 0.0.0.255 area 0
 network 192.168.108.0 0.0.0.255 area 0
 network 192.168.109.0 0.0.0.255 area 0
 network 192.168.110.0 0.0.0.255 area 0
 network 192.168.113.0 0.0.0.3 area 0
 network 192.168.115.0 0.0.0.3 area 0
!
ip classless
!
ip flow-export version 9
!
!
ip access-list extended SERVER
 permit ip any any
ip access-list extended NFS
 permit tcp host 192.168.109.1 192.168.104.0 0.0.0.255 eq 2049
 permit udp host 192.168.109.1 192.168.104.0 0.0.0.255 eq 2049
ip access-list extended IPA
 permit tcp host 192.168.109.1 192.168.96.0 0.0.15.255 eq 389
 permit tcp host 192.168.109.1 192.168.96.0 0.0.15.255 eq 636
 permit tcp host 192.168.109.1 192.168.96.0 0.0.15.255 eq www
 permit tcp host 192.168.109.1 192.168.96.0 0.0.15.255 eq 443
 permit tcp host 192.168.109.1 192.168.96.0 0.0.15.255 eq 88
 permit tcp host 192.168.109.1 192.168.96.0 0.0.15.255 eq 464
 permit tcp host 192.168.109.1 192.168.96.0 0.0.15.255 eq domain
 permit udp host 192.168.109.1 192.168.96.0 0.0.15.255 eq 88
 permit udp host 192.168.109.1 192.168.96.0 0.0.15.255 eq 464
 permit udp host 192.168.109.1 192.168.96.0 0.0.15.255 eq domain
ip access-list extended WEB
 permit tcp host 192.168.109.1 192.168.96.0 0.0.15.255 eq 8385
ip access-list extended DHCP
 permit udp host 192.168.109.1 192.168.96.0 0.0.15.255 eq bootpc
 permit udp any any eq bootps
ip access-list extended NTP
 permit udp host 192.168.109.1 192.168.96.0 0.0.15.255 eq 123
ip access-list extended SELLS
 permit ospf any any
 permit udp any any eq bootps
 permit udp host 192.168.102.255 host 192.168.109.1 eq bootpc
 permit ip 192.168.101.0 0.0.0.255 any
 permit ip 192.168.102.0 0.0.0.255 host 192.168.112.2
 permit ip 192.168.102.0 0.0.0.255 host 192.168.113.2
 permit ip 192.168.102.0 0.0.0.255 host 192.168.114.2
 permit ip 192.168.102.0 0.0.0.255 host 192.168.115.2
 permit ip 192.168.102.0 0.0.0.255 host 192.168.116.2
 permit ip 192.168.102.0 0.0.0.255 host 1.1.1.1
 permit ip 192.168.102.0 0.0.0.255 host 2.2.2.2
 permit icmp 192.168.102.0 0.0.0.255 host 192.168.109.1 echo
 permit icmp 192.168.102.0 0.0.0.255 host 192.168.109.1 echo-reply
 permit tcp 192.168.102.0 0.0.0.255 host 192.168.109.1 eq domain
 permit tcp 192.168.102.0 0.0.0.255 host 192.168.109.1 eq www
 permit udp 192.168.102.0 0.0.0.255 host 192.168.109.1 eq domain
 permit tcp 192.168.102.0 0.0.0.255 host 192.168.109.1 eq 389
 permit tcp 192.168.102.0 0.0.0.255 host 192.168.109.1 eq 636
 permit tcp 192.168.102.0 0.0.0.255 host 192.168.109.1 eq 88
 permit tcp 192.168.102.0 0.0.0.255 host 192.168.109.1 eq 464
 permit tcp 192.168.102.0 0.0.0.255 host 192.168.109.1 eq 2049
 permit tcp 192.168.102.0 0.0.0.255 host 192.168.109.1 eq 8385
 permit udp 192.168.102.0 0.0.0.255 host 192.168.109.1 eq 123
 permit udp 192.168.102.0 0.0.0.255 host 192.168.109.1 eq 88
 permit udp 192.168.102.0 0.0.0.255 host 192.168.109.1 eq 464
 permit udp 192.168.102.0 0.0.0.255 host 192.168.109.1 eq 2049
ip access-list extended STOCKS
 permit ospf any any
 permit udp any any eq bootps
 permit udp host 192.168.103.255 host 192.168.109.1 eq bootpc
 permit ip 192.168.101.0 0.0.0.255 any
 permit ip 192.168.103.0 0.0.0.255 host 192.168.112.2
 permit ip 192.168.103.0 0.0.0.255 host 192.168.113.2
 permit ip 192.168.103.0 0.0.0.255 host 192.168.114.2
 permit ip 192.168.103.0 0.0.0.255 host 192.168.115.2
 permit ip 192.168.103.0 0.0.0.255 host 192.168.116.2
 permit ip 192.168.103.0 0.0.0.255 host 1.1.1.1
 permit ip 192.168.103.0 0.0.0.255 host 2.2.2.2
 permit icmp 192.168.103.0 0.0.0.255 host 192.168.109.1 echo
 permit icmp 192.168.103.0 0.0.0.255 host 192.168.109.1 echo-reply
 permit tcp 192.168.103.0 0.0.0.255 host 192.168.109.1 eq domain
 permit tcp 192.168.103.0 0.0.0.255 host 192.168.109.1 eq www
 permit udp 192.168.103.0 0.0.0.255 host 192.168.109.1 eq domain
 permit tcp 192.168.103.0 0.0.0.255 host 192.168.109.1 eq 389
 permit tcp 192.168.103.0 0.0.0.255 host 192.168.109.1 eq 636
 permit tcp 192.168.103.0 0.0.0.255 host 192.168.109.1 eq 88
 permit tcp 192.168.103.0 0.0.0.255 host 192.168.109.1 eq 464
 permit tcp 192.168.103.0 0.0.0.255 host 192.168.109.1 eq 2049
 permit tcp 192.168.103.0 0.0.0.255 host 192.168.109.1 eq 8385
 permit udp 192.168.103.0 0.0.0.255 host 192.168.109.1 eq 123
 permit udp 192.168.103.0 0.0.0.255 host 192.168.109.1 eq 88
 permit udp 192.168.103.0 0.0.0.255 host 192.168.109.1 eq 464
 permit udp 192.168.103.0 0.0.0.255 host 192.168.109.1 eq 2049
ip access-list extended BUKH
 permit ospf any any
 permit udp any any eq bootps
 permit udp host 192.168.104.255 host 192.168.109.1 eq bootpc
 permit ip 192.168.101.0 0.0.0.255 any
 permit ip 192.168.104.0 0.0.0.255 host 192.168.112.2
 permit ip 192.168.104.0 0.0.0.255 host 192.168.113.2
 permit ip 192.168.104.0 0.0.0.255 host 192.168.114.2
 permit ip 192.168.104.0 0.0.0.255 host 192.168.115.2
 permit ip 192.168.104.0 0.0.0.255 host 192.168.116.2
 permit ip 192.168.104.0 0.0.0.255 host 1.1.1.1
 permit ip 192.168.104.0 0.0.0.255 host 2.2.2.2
 permit icmp 192.168.104.0 0.0.0.255 host 192.168.109.1 echo
 permit icmp 192.168.104.0 0.0.0.255 host 192.168.109.1 echo-reply
 permit tcp 192.168.104.0 0.0.0.255 host 192.168.109.1 eq domain
 permit tcp 192.168.104.0 0.0.0.255 host 192.168.109.1 eq www
 permit udp 192.168.104.0 0.0.0.255 host 192.168.109.1 eq domain
 permit tcp 192.168.104.0 0.0.0.255 host 192.168.109.1 eq 389
 permit tcp 192.168.104.0 0.0.0.255 host 192.168.109.1 eq 636
 permit tcp 192.168.104.0 0.0.0.255 host 192.168.109.1 eq 88
 permit tcp 192.168.104.0 0.0.0.255 host 192.168.109.1 eq 464
 permit tcp 192.168.104.0 0.0.0.255 host 192.168.109.1 eq 2049
 permit tcp 192.168.104.0 0.0.0.255 host 192.168.109.1 eq 8385
 permit udp 192.168.104.0 0.0.0.255 host 192.168.109.1 eq 123
 permit udp 192.168.104.0 0.0.0.255 host 192.168.109.1 eq 88
 permit udp 192.168.104.0 0.0.0.255 host 192.168.109.1 eq 464
 permit udp 192.168.104.0 0.0.0.255 host 192.168.109.1 eq 2049
ip access-list extended HR
 permit ospf any any
 permit udp any any eq bootps
 permit udp host 192.168.105.255 host 192.168.109.1 eq bootpc
 permit ip 192.168.101.0 0.0.0.255 any
 permit ip 192.168.105.0 0.0.0.255 host 192.168.112.2
 permit ip 192.168.105.0 0.0.0.255 host 192.168.113.2
 permit ip 192.168.105.0 0.0.0.255 host 192.168.114.2
 permit ip 192.168.105.0 0.0.0.255 host 192.168.115.2
 permit ip 192.168.105.0 0.0.0.255 host 192.168.116.2
 permit ip 192.168.105.0 0.0.0.255 host 1.1.1.1
 permit ip 192.168.105.0 0.0.0.255 host 2.2.2.2
 permit icmp 192.168.105.0 0.0.0.255 host 192.168.109.1 echo
 permit icmp 192.168.105.0 0.0.0.255 host 192.168.109.1 echo-reply
 permit tcp 192.168.105.0 0.0.0.255 host 192.168.109.1 eq domain
 permit tcp 192.168.105.0 0.0.0.255 host 192.168.109.1 eq www
 permit udp 192.168.105.0 0.0.0.255 host 192.168.109.1 eq domain
 permit tcp 192.168.105.0 0.0.0.255 host 192.168.109.1 eq 389
 permit tcp 192.168.105.0 0.0.0.255 host 192.168.109.1 eq 636
 permit tcp 192.168.105.0 0.0.0.255 host 192.168.109.1 eq 88
 permit tcp 192.168.105.0 0.0.0.255 host 192.168.109.1 eq 464
 permit tcp 192.168.105.0 0.0.0.255 host 192.168.109.1 eq 2049
 permit tcp 192.168.105.0 0.0.0.255 host 192.168.109.1 eq 8385
 permit udp 192.168.105.0 0.0.0.255 host 192.168.109.1 eq 123
 permit udp 192.168.105.0 0.0.0.255 host 192.168.109.1 eq 88
 permit udp 192.168.105.0 0.0.0.255 host 192.168.109.1 eq 464
 permit udp 192.168.105.0 0.0.0.255 host 192.168.109.1 eq 2049
ip access-list extended HEADS
 permit ospf any any
 permit udp any any eq bootps
 permit udp host 192.168.106.255 host 192.168.109.1 eq bootpc
 permit ip 192.168.101.0 0.0.0.255 any
 permit ip 192.168.106.0 0.0.0.255 host 192.168.112.2
 permit ip 192.168.106.0 0.0.0.255 host 192.168.113.2
 permit ip 192.168.106.0 0.0.0.255 host 192.168.114.2
 permit ip 192.168.106.0 0.0.0.255 host 192.168.115.2
 permit ip 192.168.106.0 0.0.0.255 host 192.168.116.2
 permit ip 192.168.106.0 0.0.0.255 host 1.1.1.1
 permit ip 192.168.106.0 0.0.0.255 host 2.2.2.2
 permit icmp 192.168.106.0 0.0.0.255 host 192.168.109.1 echo
 permit icmp 192.168.106.0 0.0.0.255 host 192.168.109.1 echo-reply
 permit tcp 192.168.106.0 0.0.0.255 host 192.168.109.1 eq domain
 permit tcp 192.168.106.0 0.0.0.255 host 192.168.109.1 eq www
 permit udp 192.168.106.0 0.0.0.255 host 192.168.109.1 eq domain
 permit tcp 192.168.106.0 0.0.0.255 host 192.168.109.1 eq 389
 permit tcp 192.168.106.0 0.0.0.255 host 192.168.109.1 eq 636
 permit tcp 192.168.106.0 0.0.0.255 host 192.168.109.1 eq 88
 permit tcp 192.168.106.0 0.0.0.255 host 192.168.109.1 eq 464
 permit tcp 192.168.106.0 0.0.0.255 host 192.168.109.1 eq 2049
 permit tcp 192.168.106.0 0.0.0.255 host 192.168.109.1 eq 8385
 permit udp 192.168.106.0 0.0.0.255 host 192.168.109.1 eq 123
 permit udp 192.168.106.0 0.0.0.255 host 192.168.109.1 eq 88
 permit udp 192.168.106.0 0.0.0.255 host 192.168.109.1 eq 464
 permit udp 192.168.106.0 0.0.0.255 host 192.168.109.1 eq 2049
ip access-list extended AHO
 permit ospf any any
 permit udp any any eq bootps
 permit udp host 192.168.107.255 host 192.168.109.1 eq bootpc
 permit ip 192.168.101.0 0.0.0.255 any
 permit ip 192.168.107.0 0.0.0.255 host 192.168.112.2
 permit ip 192.168.107.0 0.0.0.255 host 192.168.113.2
 permit ip 192.168.107.0 0.0.0.255 host 192.168.114.2
 permit ip 192.168.107.0 0.0.0.255 host 192.168.115.2
 permit ip 192.168.107.0 0.0.0.255 host 192.168.116.2
 permit ip 192.168.107.0 0.0.0.255 host 1.1.1.1
 permit ip 192.168.107.0 0.0.0.255 host 2.2.2.2
 permit icmp 192.168.107.0 0.0.0.255 host 192.168.109.1 echo
 permit icmp 192.168.107.0 0.0.0.255 host 192.168.109.1 echo-reply
 permit tcp 192.168.107.0 0.0.0.255 host 192.168.109.1 eq domain
 permit tcp 192.168.107.0 0.0.0.255 host 192.168.109.1 eq www
 permit udp 192.168.107.0 0.0.0.255 host 192.168.109.1 eq domain
 permit tcp 192.168.107.0 0.0.0.255 host 192.168.109.1 eq 389
 permit tcp 192.168.107.0 0.0.0.255 host 192.168.109.1 eq 636
 permit tcp 192.168.107.0 0.0.0.255 host 192.168.109.1 eq 88
 permit tcp 192.168.107.0 0.0.0.255 host 192.168.109.1 eq 464
 permit tcp 192.168.107.0 0.0.0.255 host 192.168.109.1 eq 2049
 permit tcp 192.168.107.0 0.0.0.255 host 192.168.109.1 eq 8385
 permit udp 192.168.107.0 0.0.0.255 host 192.168.109.1 eq 123
 permit udp 192.168.107.0 0.0.0.255 host 192.168.109.1 eq 88
 permit udp 192.168.107.0 0.0.0.255 host 192.168.109.1 eq 464
 permit udp 192.168.107.0 0.0.0.255 host 192.168.109.1 eq 2049
ip access-list extended SEC
 permit ospf any any
 permit udp any any eq bootps
 permit udp host 192.168.108.255 host 192.168.109.1 eq bootpc
 permit ip 192.168.101.0 0.0.0.255 any
 permit ip 192.168.108.0 0.0.0.255 host 192.168.112.2
 permit ip 192.168.108.0 0.0.0.255 host 192.168.113.2
 permit ip 192.168.108.0 0.0.0.255 host 192.168.114.2
 permit ip 192.168.108.0 0.0.0.255 host 192.168.115.2
 permit ip 192.168.108.0 0.0.0.255 host 192.168.116.2
 permit ip 192.168.108.0 0.0.0.255 host 1.1.1.1
 permit ip 192.168.108.0 0.0.0.255 host 2.2.2.2
 permit icmp 192.168.108.0 0.0.0.255 host 192.168.109.1 echo
 permit icmp 192.168.108.0 0.0.0.255 host 192.168.109.1 echo-reply
 permit tcp 192.168.108.0 0.0.0.255 host 192.168.109.1 eq domain
 permit tcp 192.168.108.0 0.0.0.255 host 192.168.109.1 eq www
 permit udp 192.168.108.0 0.0.0.255 host 192.168.109.1 eq domain
 permit tcp 192.168.108.0 0.0.0.255 host 192.168.109.1 eq 389
 permit tcp 192.168.108.0 0.0.0.255 host 192.168.109.1 eq 636
 permit tcp 192.168.108.0 0.0.0.255 host 192.168.109.1 eq 88
 permit tcp 192.168.108.0 0.0.0.255 host 192.168.109.1 eq 464
 permit tcp 192.168.108.0 0.0.0.255 host 192.168.109.1 eq 2049
 permit tcp 192.168.108.0 0.0.0.255 host 192.168.109.1 eq 8385
 permit udp 192.168.108.0 0.0.0.255 host 192.168.109.1 eq 123
 permit udp 192.168.108.0 0.0.0.255 host 192.168.109.1 eq 88
 permit udp 192.168.108.0 0.0.0.255 host 192.168.109.1 eq 464
 permit udp 192.168.108.0 0.0.0.255 host 192.168.109.1 eq 2049
ip access-list extended IT
 permit udp any any eq bootps
 permit udp host 192.168.101.255 host 192.168.109.1 eq bootpc
 permit ip 192.168.101.0 0.0.0.255 any
 permit ospf any any
!
banner motd #
Unauthorized access is prohibited
#
!
!
!
!
line con 0
 password 7 0822455D0A16
 login
!
line aux 0
!
line vty 0 4
 password 7 0822455D0A16
 login local
 transport input ssh
line vty 5 14
 password 7 0822455D0A16
 login local
 transport input ssh
line vty 15
 login local
 transport input ssh
!
!
!
!
end

\end{lstlisting}

\begin{lstlisting}[caption=Конфигурация устройства SW\_1\_L2\_IVANOV\label{list:conf_sw1_l2}]
!
version 15.0
no service timestamps log datetime msec
no service timestamps debug datetime msec
service password-encryption
!
hostname SW_1_L2_IVANOV
!
enable password 7 0822455D0A16
!
!
!
ip ssh authentication-retries 2
ip ssh time-out 60
no ip domain-lookup
ip domain-name sw.ivanov.test
!
username SSHAdmin privilege 15 password 7 0822455D0A16
!
!
!
spanning-tree mode pvst
spanning-tree extend system-id
!
interface FastEthernet0/1
 switchport access vlan 101
 switchport mode access
!
interface FastEthernet0/2
 switchport access vlan 101
 switchport mode access
!
interface FastEthernet0/3
 switchport access vlan 101
 switchport mode access
!
interface FastEthernet0/4
 switchport access vlan 101
 switchport mode access
!
interface FastEthernet0/5
 switchport access vlan 101
 switchport mode access
!
interface FastEthernet0/6
 switchport access vlan 101
 switchport mode access
!
interface FastEthernet0/7
 switchport access vlan 101
 switchport mode access
!
interface FastEthernet0/8
 switchport access vlan 101
 switchport mode access
!
interface FastEthernet0/9
 switchport access vlan 101
 switchport mode access
!
interface FastEthernet0/10
 switchport access vlan 101
 switchport mode access
!
interface FastEthernet0/11
 switchport access vlan 101
 switchport mode access
!
interface FastEthernet0/12
 switchport access vlan 101
 switchport mode access
!
interface FastEthernet0/13
 switchport access vlan 101
 switchport mode access
!
interface FastEthernet0/14
 switchport access vlan 101
 switchport mode access
!
interface FastEthernet0/15
 switchport access vlan 101
 switchport mode access
!
interface FastEthernet0/16
 switchport access vlan 101
 switchport mode access
!
interface FastEthernet0/17
 switchport access vlan 101
 switchport mode access
!
interface FastEthernet0/18
 switchport access vlan 101
 switchport mode access
!
interface FastEthernet0/19
 switchport access vlan 101
 switchport mode access
!
interface FastEthernet0/20
 switchport access vlan 101
 switchport mode access
!
interface FastEthernet0/21
 switchport access vlan 101
 switchport mode access
!
interface FastEthernet0/22
 switchport access vlan 101
 switchport mode access
!
interface FastEthernet0/23
 switchport access vlan 101
 switchport mode access
!
interface FastEthernet0/24
 switchport access vlan 101
 switchport mode access
!
interface GigabitEthernet0/1
 switchport trunk allowed vlan 101-115
 switchport mode trunk
!
interface GigabitEthernet0/2
 switchport trunk allowed vlan 101-115
 switchport mode trunk
!
interface Vlan1
 no ip address
 shutdown
!
interface Vlan110
 ip address 192.168.110.1 255.255.255.0
!
ip default-gateway 192.168.110.254
!
banner motd #
Unauthorized access is prohibited!
#
!
!
!
line con 0
 password 7 0822455D0A16
 login
!
line vty 0 4
 password 7 0822455D0A16
 login local
 transport input ssh
line vty 5 15
 password 7 0822455D0A16
 login local
 transport input ssh
!
!
!
!
end

\end{lstlisting}

\begin{lstlisting}[caption=Конфигурация устройства SW\_2\_L2\_IVANOV\label{list:conf_sw2_l2}]
!
version 15.0
no service timestamps log datetime msec
no service timestamps debug datetime msec
service password-encryption
!
hostname SW_2_L2_IVANOV
!
enable password 7 0822455D0A16
!
!
!
ip ssh authentication-retries 2
ip ssh time-out 60
no ip domain-lookup
ip domain-name sw.ivanov.test
!
username SSHAdmin privilege 15 password 7 0822455D0A16
!
!
!
spanning-tree mode pvst
spanning-tree extend system-id
!
interface FastEthernet0/1
 switchport access vlan 104
 switchport mode access
!
interface FastEthernet0/2
 switchport access vlan 104
 switchport mode access
!
interface FastEthernet0/3
 switchport access vlan 104
 switchport mode access
!
interface FastEthernet0/4
 switchport access vlan 104
 switchport mode access
!
interface FastEthernet0/5
 switchport access vlan 104
 switchport mode access
!
interface FastEthernet0/6
 switchport access vlan 104
 switchport mode access
!
interface FastEthernet0/7
 switchport access vlan 104
 switchport mode access
!
interface FastEthernet0/8
 switchport access vlan 104
 switchport mode access
!
interface FastEthernet0/9
 switchport access vlan 104
 switchport mode access
!
interface FastEthernet0/10
 switchport access vlan 104
 switchport mode access
!
interface FastEthernet0/11
 switchport access vlan 104
 switchport mode access
!
interface FastEthernet0/12
 switchport access vlan 107
 switchport mode access
!
interface FastEthernet0/13
 switchport access vlan 108
 switchport mode access
!
interface FastEthernet0/14
!
interface FastEthernet0/15
!
interface FastEthernet0/16
!
interface FastEthernet0/17
!
interface FastEthernet0/18
!
interface FastEthernet0/19
!
interface FastEthernet0/20
!
interface FastEthernet0/21
!
interface FastEthernet0/22
!
interface FastEthernet0/23
!
interface FastEthernet0/24
!
interface GigabitEthernet0/1
 switchport trunk allowed vlan 101-111
 switchport mode trunk
!
interface GigabitEthernet0/2
 switchport trunk allowed vlan 101-111
 switchport mode trunk
!
interface Vlan1
 no ip address
 shutdown
!
interface Vlan110
 ip address 192.168.110.2 255.255.255.0
!
ip default-gateway 192.168.110.254
!
banner motd #
Unauthorized access is prohibited
#
!
!
!
line con 0
 password 7 0822455D0A16
 login
!
line vty 0 4
 password 7 0822455D0A16
 login local
 transport input ssh
line vty 5 14
 password 7 0822455D0A16
 login local
 transport input ssh
line vty 15
 login local
 transport input ssh
!
!
!
!
end

\end{lstlisting}

\begin{lstlisting}[caption=Конфигурация устройства SW\_3\_L2\_IVANOV\label{list:conf_sw3_l2}]
!
version 16.3.2
no service timestamps log datetime msec
no service timestamps debug datetime msec
service password-encryption
!
hostname SW_3_L2_IVANOV
!
!
enable password 7 0822455D0A16
!
!
!
!
!
!
no ip cef
no ipv6 cef
!
!
!
username SSHAdmin privilege 15 password 7 0822455D0A16
!
!
!
!
!
!
!
!
!
!
ip ssh authentication-retries 2
ip ssh time-out 60
no ip domain-lookup
ip domain-name sw.ivanov.test
!
!
spanning-tree mode pvst
!
!
!
!
!
!
interface Port-channel1
 switchport trunk allowed vlan 101-111
 switchport mode trunk
!
interface Port-channel2
 switchport trunk allowed vlan 101-111
 switchport mode trunk
!
interface GigabitEthernet1/0/1
 switchport access vlan 105
 switchport mode access
!
interface GigabitEthernet1/0/2
 switchport access vlan 105
 switchport mode access
!
interface GigabitEthernet1/0/3
 switchport access vlan 105
 switchport mode access
!
interface GigabitEthernet1/0/4
 switchport access vlan 105
 switchport mode access
!
interface GigabitEthernet1/0/5
 switchport access vlan 105
 switchport mode access
!
interface GigabitEthernet1/0/6
 switchport access vlan 105
 switchport mode access
!
interface GigabitEthernet1/0/7
 switchport access vlan 105
 switchport mode access
!
interface GigabitEthernet1/0/8
 switchport access vlan 105
 switchport mode access
!
interface GigabitEthernet1/0/9
 switchport access vlan 105
 switchport mode access
!
interface GigabitEthernet1/0/10
 switchport access vlan 105
 switchport mode access
!
interface GigabitEthernet1/0/11
 switchport access vlan 105
 switchport mode access
!
interface GigabitEthernet1/0/12
 switchport access vlan 101
 switchport mode access
!
interface GigabitEthernet1/0/13
 switchport access vlan 101
 switchport mode access
!
interface GigabitEthernet1/0/14
 switchport access vlan 101
 switchport mode access
!
interface GigabitEthernet1/0/15
 switchport access vlan 106
 switchport mode access
!
interface GigabitEthernet1/0/16
 switchport access vlan 106
 switchport mode access
!
interface GigabitEthernet1/0/17
 switchport access vlan 106
 switchport mode access
!
interface GigabitEthernet1/0/18
 switchport trunk allowed vlan 101-111
 switchport mode trunk
 channel-protocol lacp
 channel-group 1 mode active
!
interface GigabitEthernet1/0/19
 switchport trunk allowed vlan 101-111
 switchport mode trunk
 channel-protocol lacp
 channel-group 1 mode active
!
interface GigabitEthernet1/0/20
 switchport trunk allowed vlan 101-111
 switchport mode trunk
 channel-protocol lacp
 channel-group 1 mode active
!
interface GigabitEthernet1/0/21
 switchport trunk allowed vlan 101-111
 switchport mode trunk
 channel-protocol lacp
 channel-group 2 mode active
!
interface GigabitEthernet1/0/22
 switchport trunk allowed vlan 101-111
 switchport mode trunk
 channel-protocol lacp
 channel-group 2 mode active
!
interface GigabitEthernet1/0/23
 switchport trunk allowed vlan 101-111
 switchport mode trunk
 channel-protocol lacp
 channel-group 2 mode active
!
interface GigabitEthernet1/0/24
!
interface GigabitEthernet1/1/1
!
interface GigabitEthernet1/1/2
!
interface GigabitEthernet1/1/3
!
interface GigabitEthernet1/1/4
!
interface Vlan1
 no ip address
 shutdown
!
interface Vlan110
 mac-address 0060.70da.de01
 ip address 192.168.110.3 255.255.255.0
!
ip default-gateway 192.168.110.254
ip classless
!
ip flow-export version 9
!
!
!
banner motd #
Unauthorized access is prohibited
#
!
!
!
!
line con 0
 password 7 0822455D0A16
 login
!
line aux 0
!
line vty 0 4
 password 7 0822455D0A16
 login local
 transport input ssh
line vty 5 14
 password 7 0822455D0A16
 login local
 transport input ssh
line vty 15
 login local
 transport input ssh
!
!
!
!
end

\end{lstlisting}

\begin{lstlisting}[caption=Конфигурация устройства SW\_4\_L2\_IVANOV\label{list:conf_sw4_l2}]
!
version 15.0
no service timestamps log datetime msec
no service timestamps debug datetime msec
service password-encryption
!
hostname SW_4_L2_IVANOV
!
enable password 7 0822455D0A16
!
!
!
ip ssh authentication-retries 2
ip ssh time-out 60
no ip domain-lookup
ip domain-name sw.ivanov.test
!
username SSHAdmin privilege 15 password 7 0822455D0A16
!
!
!
spanning-tree mode pvst
spanning-tree extend system-id
!
interface FastEthernet0/1
 switchport access vlan 102
 switchport mode access
!
interface FastEthernet0/2
 switchport access vlan 102
 switchport mode access
!
interface FastEthernet0/3
 switchport access vlan 102
 switchport mode access
!
interface FastEthernet0/4
 switchport access vlan 102
 switchport mode access
!
interface FastEthernet0/5
 switchport access vlan 102
 switchport mode access
!
interface FastEthernet0/6
 switchport access vlan 102
 switchport mode access
!
interface FastEthernet0/7
 switchport access vlan 102
 switchport mode access
!
interface FastEthernet0/8
 switchport access vlan 102
 switchport mode access
!
interface FastEthernet0/9
 switchport access vlan 102
 switchport mode access
!
interface FastEthernet0/10
 switchport access vlan 102
 switchport mode access
!
interface FastEthernet0/11
 switchport access vlan 102
 switchport mode access
!
interface FastEthernet0/12
 switchport access vlan 103
 switchport mode access
!
interface FastEthernet0/13
 switchport access vlan 103
 switchport mode access
!
interface FastEthernet0/14
 switchport access vlan 103
 switchport mode access
!
interface FastEthernet0/15
 switchport access vlan 103
 switchport mode access
!
interface FastEthernet0/16
 switchport access vlan 103
 switchport mode access
!
interface FastEthernet0/17
 switchport access vlan 103
 switchport mode access
!
interface FastEthernet0/18
 switchport access vlan 103
 switchport mode access
!
interface FastEthernet0/19
 switchport access vlan 103
 switchport mode access
!
interface FastEthernet0/20
 switchport access vlan 103
 switchport mode access
!
interface FastEthernet0/21
 switchport access vlan 103
 switchport mode access
!
interface FastEthernet0/22
 switchport access vlan 103
 switchport mode access
!
interface FastEthernet0/23
!
interface FastEthernet0/24
!
interface GigabitEthernet0/1
 switchport trunk allowed vlan 101-111
 switchport mode trunk
!
interface GigabitEthernet0/2
 switchport trunk allowed vlan 101-111
 switchport mode trunk
!
interface Vlan1
 no ip address
 shutdown
!
interface Vlan110
 ip address 192.168.110.4 255.255.255.0
!
ip default-gateway 192.168.110.254
!
banner motd #
Unauthorized access is prohibited
#
!
!
!
line con 0
 password 7 0822455D0A16
 login
!
line vty 0 4
 password 7 0822455D0A16
 login local
 transport input ssh
line vty 5 14
 password 7 0822455D0A16
 login local
 transport input ssh
line vty 15
 login local
 transport input ssh
!
!
!
!
end

\end{lstlisting}

\begin{lstlisting}[caption=Конфигурация устройства SW\_5\_L2\_IVANOV\label{list:conf_sw5_l2}]
!
version 16.3.2
no service timestamps log datetime msec
no service timestamps debug datetime msec
service password-encryption
!
hostname SW_5_L2_IVANOV
!
!
enable password 7 0822455D0A16
!
!
!
!
!
!
no ip cef
no ipv6 cef
!
!
!
username SSHAdmin privilege 15 password 7 0822455D0A16
!
!
!
!
!
!
!
!
!
!
ip ssh authentication-retries 2
ip ssh time-out 60
no ip domain-lookup
ip domain-name sw.ivanov.test
!
!
spanning-tree mode pvst
!
!
!
!
!
!
interface GigabitEthernet1/0/1
 switchport access vlan 109
 switchport trunk allowed vlan 105
 switchport mode access
!
interface GigabitEthernet1/0/2
 switchport trunk allowed vlan 101-111
 switchport mode trunk
!
interface GigabitEthernet1/0/3
 switchport trunk allowed vlan 101-111
 switchport mode trunk
!
interface GigabitEthernet1/0/4
!
interface GigabitEthernet1/0/5
!
interface GigabitEthernet1/0/6
!
interface GigabitEthernet1/0/7
!
interface GigabitEthernet1/0/8
!
interface GigabitEthernet1/0/9
!
interface GigabitEthernet1/0/10
!
interface GigabitEthernet1/0/11
!
interface GigabitEthernet1/0/12
!
interface GigabitEthernet1/0/13
!
interface GigabitEthernet1/0/14
!
interface GigabitEthernet1/0/15
!
interface GigabitEthernet1/0/16
!
interface GigabitEthernet1/0/17
!
interface GigabitEthernet1/0/18
!
interface GigabitEthernet1/0/19
!
interface GigabitEthernet1/0/20
!
interface GigabitEthernet1/0/21
!
interface GigabitEthernet1/0/22
!
interface GigabitEthernet1/0/23
!
interface GigabitEthernet1/0/24
!
interface GigabitEthernet1/1/1
!
interface GigabitEthernet1/1/2
!
interface GigabitEthernet1/1/3
!
interface GigabitEthernet1/1/4
!
interface Vlan1
 no ip address
 shutdown
!
interface Vlan110
 mac-address 0001.63d8.b301
 ip address 192.168.110.5 255.255.255.0
!
ip default-gateway 192.168.110.254
ip classless
!
ip flow-export version 9
!
!
!
banner motd #
Unauthorized access is prohibited
#
!
!
!
!
line con 0
 password 7 0822455D0A16
 login
!
line aux 0
!
line vty 0 4
 password 7 0822455D0A16
 login local
 transport input ssh
line vty 5 14
 password 7 0822455D0A16
 login local
 transport input ssh
line vty 15
 login local
 transport input ssh
!
!
!
!
end
\end{lstlisting}

\begin{figure}[H]
  \centering
  \includegraphics[width=\textwidth]{config_dhcp}
  \caption{Конфигурация DHCP-сервера в Cisco Packet Tracer}
  \label{fig:config_dhcp}
\end{figure}

\section{Тестирование работоспособности сети}
\label{apx:test}

\begin{figure}[H]
  \centering
  \includegraphics[width=\textwidth]{cisco_topology}
  \caption{Смоделированная топология в Cisco Packet Tracer}
  \label{fig:cisco_topology}
\end{figure}

\begin{figure}[H]
  \centering
  \includegraphics[width=\textwidth]{cisco_test_ping}
  \caption{Проверка соединения между различными устройствами}
  \label{fig:cisco_test_ping}
\end{figure}

\begin{figure}[H]
  \centering
  \includegraphics[width=\textwidth]{cisco_test_web}
  \caption{Проверка работоспособности веб-службы и службы доменных имен}
  \label{fig:cisco_test_web}
\end{figure}

\begin{figure}[H]
  \centering
  \includegraphics[width=\textwidth]{cisco_test_dhcp}
  \caption{Проверка работоспособности службы динамического конфигурирования хостов}
  \label{fig:cisco_test_dhcp}
\end{figure}

\end{appendices}

\end{document}